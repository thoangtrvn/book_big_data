\chapter{Web Search Engines}
\label{chap:Web_SearchEngines}

\url{http://en.wikipedia.org/wiki/List_of_search_engines}

Building Internet-scale search engines requires huge amounts of data and
therefore large numbers of machines to process it. Mike Cafarella and Doug
Cutting estimated a system supporting a 1-billion-page index would cost around
half a million dollars in hardware, with a monthly running cost of \$30,000.


Yahoo! Search consists of four primary components
\begin{enumerate}
  \item Crawler: download the pages (from web servers) - Chap.\ref{chap:WebCrawling}
  
  \item WebMap: build a (huge) graphs of this known web (i.e. the linking
  between pages)
  
  The graph being built for the Internet can reach $10^{12}$ edges (each
  representing a web link), and $10^{11}$ nodes (each representing a distinct
  URL).  Creating and analyzing such a large graph requires a large number of
  computers running for many days. 
  
  \item Indexer: build a reverse index to the best pages
  \item Runtime: answer uer's queries.
\end{enumerate}

\section{Nutch (web search engine + web crawler)}
\label{sec:Nutch_websearch_engine}

Nutch is an open-source web search engine based on Lucene
(Sect.\ref{sec:Lucene}) by two authors (Doug Cutting, Mike Cafarella).
The fetcher ("robot" or "web crawler") has been written from scratch
specifically for this project. However, they soon realized that their
architecture wouldn't scale to the billions of pages on the Web when they
started in 2002. Then, when a paper describing Google's Distributed FS called
GFS was published in 2003.

GFS, or something like it, would solve their storage needs for the very large
files generated as a part of the web crawl and indexing process. In particular,
GFS would free up time being spent on administrative tasks such as managing
storage nodes. In 2004, they set about writing an open source implementation,
the Nutch Distributed Filesystem (NDFS).   

Again, with the paper describing MapReduce programming model from Google
(Sect.\ref{sec:MapReduce}), Nutch added an implementation for MapReduce
programming model and porting the existing algorithm from Nutch to using
MapReduce and NDFS.


\section{Bing Search API}

We can use the results returned by Bing Search API.
Microsoft provides RESTful APIs which can be accessed from PHP, Microsoft .NET
and C\#, or wrappers in Python (bingpy - Sect.\ref{sec:bingpy}).


\subsection{BingPy}
\label{sec:bingpy}

\url{https://pypi.python.org/pypi/bingpy/0.0.1}

\begin{verbatim}
from bingpy import WebSearch
web = WebSearch("YOUR_API_KEY")
pages = web.search("kyoto", 20)
for page in pages:
    print page.title
\end{verbatim}

Bing key or account key are used by applications to access your Microsoft Azure
Marketplace dataset subscriptions. Do not share your account keys with other users.

To get Bing Key: \url{http://www.articlekevo.com/get-bing-api-key/}
% :
% \begin{enumerate}
%   \item A distributed file system (Hadoop Distributed File System -
%   Sect.\ref{sec:HadoopFS})
%     
%   \item A MapReduce facility which become Apache Hadoop project.
% \end{enumerate}
