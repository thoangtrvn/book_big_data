\chapter{Distributed Filesystem}
\label{chap:distributed_filesystem}

A brief information is given in Sect.\ref{sec:distributed_filesystem}. Most of
the open-source distributed filesystem have been implemented for Linux.
In Hadoop, the default file system is HDFS; yet we can also use other options
\begin{itemize}
  \item open-source: GlusterFS, Quantcast File System
  \item commercial options: Isilon OneFS, Netapp
  \item cloud-based storage system: S3 (Amazon's Simple Storage System)
\end{itemize}

\section{Google FS (Colossus)}
\label{sec:GoogleFS}


The new version of Google FS is named Colossus. 


\section{Hadoop FS (HDFS)}
\label{sec:HadoopFS} 

HDFS is a highly fault-tolerant distributed file system, developed as
part of Hadoop (Sect.\ref{chap:Hadoop}).

Like Hadoop in general, HDFS is designed to be deployed on low-cost hardware.
While most block-structured file systems use a block size on the order of 4 or 8
KB; Hadoop splits files into large blocks (default 64MB or 128MB) and distribute
the blocks among the nodes in the cluster. This allows HDFS to decrease the
amount of metadata storage required per file (the list of blocks per file will
be smaller as the size of individual blocks increases).  A wrapper to HDFS is
HBase (Chapter \ref{chap:HBase}).

Because of the relatively low amount of metadata per file (it only tracks file
names, permissions, and the locations of each block of each file), all of this
information can be stored in the main memory of the NameNode machine, allowing
fast access to the metadata.

HDFS allows for fast streaming reads of data, by keeping large amounts of data
sequentially laid out on the disk. The consequence of this decision is that HDFS
expects to have very large files, and expects them to be read sequentially.
Thus, EXT4 filesystem is considered a good option.

You cannot interact with HDFS-stored files using ordinary Linux file
modification tools (e.g., ls, cp, mv, etc). Because HDFS stores files as a set
of large blocks across several machines, these files are not part of the
ordinary file system. HDFS runs in a separate namespace, isolated from the
contents of your local files.
The files inside HDFS (or more accurately: the blocks that make them up) are
stored in a particular directory managed by the DataNode service, but the files
will named only with block ids.

HDFS does come with its own utilities for file management, which act very
similar to these familiar tools.
To open a file, a client contacts the NameNode and retrieves a list of locations
for the blocks that comprise the file.
Clients then read file data directly from the DataNode servers, possibly in
parallel. The NameNode is not directly involved in this bulk data transfer,
keeping its overhead to a minimum. 

Just as with a standard file system, Hadoop allows for storage of data in any
format, whether it's text, binary, images, etc. Hadoop also provides built-in
support for a number of formats optimized for Hadoop storage and processing,
text files, {\bf SequenceFile}, or more complex (but also more functionally
rich) options like Avro and Parquet. It's possible to create your own custom
file format in Hadoop as well.

There are several Hadoop specific file formats that were specifically created to
work well with MapReduce. These Hadoop specific file formats include file based
data structures such as sequence files, serialization formats like Avro, and
columnar formats such as RCFiles and Parquet.   These file formats have
differing strengths and weaknesses, but all share the following characteristics
that are important for Hadoop applications: 
\begin{itemize}
  \item splittable compression
  \item agnostic compression: the file can be compressed with any compression
  method, without readers having to know the codec.
\end{itemize}
The SequenceFile format is one of most commonly used file based format in
Hadoop, but other file based formats are available such as MapFiles, SetFiles,
ArrayFiles, and BloomMapFiles. Since these formats were specifically designed to
work with MapReduce, they offer a high level of integration for all forms of
MapReduce jobs, including those run via Pig and Hive.    
\url{https://www.safaribooksonline.com/library/view/hadoop-application-architectures/9781491910313/ch01.html}

The first step is to copy data onto the HDFS, i.e. the {\bf ingestion process}.
The following list describes which tools to use for data from a particular
source or data format
\begin{itemize}
  \item \ldots : Talend
  \url{http://en.wikipedia.org/wiki/Talend}
  
  
  \item RDBMS: use Apache Sqoop or HBase, or MapR M7 tables
  
  Example: 1 TB of data take 3hr of ingestion using Sqoop.
  
  HBase and M7 have identical API and differ only in what kind of latencies you
  can expect
  
  \item Bulk file: use FTP and then ETL tools to load into HDFS
  
  ETL tool are too generic for the actual cases at hand. Usually, the ETL tools
  provide a way of running custom code as part of the work flow, which is a
  solution. 
  
  \item Realtime Data: using WebService + Flume (log aggregation, near
  real-time loading - near real-time means latency about less than half a day)
  \item Realtime data: (continuous stream ingestion) Storm.
  
   Flume is good at transport and some light enrichment / decoration. Storm can
  actually do real-time heavy lifting.
  
  \item (not using HDFS, but use NFS): Customers of MapR's Hadoop distribution
  typically utilize NFS to ingest data (and subsequently make it available to
  other parts of the organization, either in a Hadoop context or not). They
  often find NFS easier to use than Flume, legacy ETL/ELT processes and the
  like. (Note that I work for MapR).
\end{itemize}
The choice depending on the data source at latency requirements.
\url{https://www.linkedin.com/groups/Data-Ingestion-Into-Hadoop-3638279.S.199779955}

Data stored in HDFS are just bytes, i.e. there is no datatypes. 
To access these data, there are several options:
\begin{enumerate}
  \item MapReduce jobs: write your own Java/C/C++ code in MapReduce jobs
  (Chap.\ref{chap:Hadoop})
  
  \item HBase's API (in Java, REST, Thrift): (Chap.\ref{chap:HBase})
  \item Move data from HDFS to relational database systems: To move data between
  HDFS and relational database, we can use Apache Sqoop.
\end{enumerate}
 


\subsection{text files (CSV, XML)}

A very common use of Hadoop is the storage and analysis of logs such as web logs
and server logs. Such text data, of course, also come in many other forms: CSV
files, or unstructured data such as emails, etc 

The data stored in HDFS can be either {\it text files} or {\it sequence files}
in the form of (key,value) pairs. Sequence files are appropriate for situations
where you want to store the keys and their corresponding values. In text files,
we can do that but we have to parse each line. The sequence file can be
compressed and still be splittable, i.e. better workload. We cannot split a
compressed text file, unless we use a splittable compresion format. Thirdly, a
sequence file can be approached as a binary files, i.e. more storage efficient;
while in text files, double values will be stored as a number of characters,
i.e. large storage overhead.

For example, when storing 1234 in a text file and using it as an integer,
requires a String to Integer conversion during read, and vice-versa during
writing. This overhead adds up when you do a lot of such conversions. 

A more specialized form of text files are structured formats such as XML and
JSON. These types of formats can present special challenges with Hadoop since
splitting XML and JSON files for processing is tricky, and Hadoop does not
provide a built-in InputFormat for either of these formats. JSON presents even
greater challenges than XML, since there are no tokens to mark the beginning or
end of a record. In the case of these formats, you have a couple of options:     
\begin{itemize}
  \item Avro: as a container format, Transforming the data into Avro can provide
  a compact and efficient way to store and process the data.
  
  \item Pig: use a library designed for processing XML or JSON files, e.g.
  XMLLoader in PiggyBank library; or with JSON using LzoJsonInputFormat in
  Elephant Bird project. 
  
\end{itemize}

\subsection{Binary format}

For most cases of storing and processing binary files in Hadoop, and the file
size is small, using a container format such as SequenceFile is preferred
(Sect.\ref{sec:SequenceFile}). If the splittable unit of binary data is larger
than 64 MB, you may consider putting the data in it's own file, without using a
container format.


\subsection{SequenceFile}
\label{sec:SequenceFile}


HDFS works well with large files, as every 'map' task processes a block of data
at a time, which is only efficient if the block of data is large enough.
To solve the problem of storing small files in HDFS, {\bf SequenceFile} are
used as a container to store small files. It is a flat file consisting of binary
key/value pairs. Internally, the temporary output of ``map'' operator are also
in SequenceFile. 

Hadoop's SequenceFile is a persistent data structure for binary key-value pairs.
Unlike other B-tree structure, we cannot do key editting, adding or removing it.
It means that the file is append-only. 

There are 3 different SequenceFile formats
\begin{itemize}
  \item uncompressed key/value records
  \item record compressed key/value record - only 'value' are compressed here
  \item block compressed keyvalue record - both 'key' and 'value' are collected
  in 'blocks' separately and compressed. The size of 'block' is configurable. 
\end{itemize}
\url{http://wiki.apache.org/hadoop/SequenceFile}


So, user may need to write a custom reader to interpret the data to tell Hadoop
which one is key, which one is value. 

\subsection{Serialization formats (Writables, Avro, Protocol Buffers, Thrift)}
\label{sec:Serialization_hadoop}

Serialization (Sect.\ref{sec:serialization}) is core to a distributed processing
system such as Hadoop, since it allows data to be converted into a format that
can be efficiently stored as well as transferred across a network connection. 
The main serialization format utilized by Hadoop is Writables. There are,
however, other serialization frameworks seeing increased use within the Hadoop
ecosystem, including Thrift, Protocol Buffers, and Avro.  Of these, Avro is the
best suited, since it was specifically created to address limitations of Hadoop
Writables.

While Avro defines a small number of primitive types such as boolean, int,
float, and string, it also supports supports complex types such as array, map, and enum.

\subsection{columnar formats (RCFile, ORC, Parquet)}

More recently, a number of databases have introduced columnar storage, which
provides several benefits over earlier row-oriented systems. Not surprisingly,
columnar file formats are also being utilized for Hadoop applications. Columnar
file formats supported on Hadoop include the RCFile format  which has been
popular for some time as a Hive format, as well as newer formats such as ORC and
Parquet which are described below. 



\section{GSS}
\label{sec:GSS}


\section{IBM GPFS}

GPFS is built on the solid foundation of GSS (Sect.\ref{sec:GSS})

It can be deployed in shared-disk or shared-nothing distributed parallel modes.
It is used by many of the world's largest commercial companies, as well as some
of the supercomputers on the Top 500 List. GPFS provides concurrent high-speed
file access to applications executing on multiple nodes of clusters.

To achieve high throughput to a single large file, data
must be striped across multiple disks.

\subsection{GPFS 3.1}

Starting with GPFS 3.1, the structural limit on the maximum number of disks in
a file system increased from 2048 to 4096 (each 1TB); however, IBM Spectrum
Scale still enforces the original limit of 2048 

\section{GPFS Native RAID (declustered RAID): software implementation
micro-RAID}

With micro-RAID,  RAID is done at a block level as opposed to a disk level. 
This generally means that the cost of rebuilds can be reduced and the time to
get back to a protected level can be shortened. 

NOTE: As disks continue to get larger, conventional RAID implementations
struggle and you can be looking at hours if not days to get back to a protected
state.







\section{OpenAFS}
\label{sec:OpenAFS}

\url{http://www.openafs.org/}

\section{Data ingestion}
\label{sec:data_ingestion}

Data ingestion depends on
\begin{itemize}
  \item CPU 
  \item RAM
  \item Network card
  \item Harddrive
  \item Network bandwidth
\end{itemize}
\url{http://www.slideshare.net/Hadoop_Summit/mc-cuch-zhurakouskyjune261210pmroom212}

\section{GlusterFS}

GlusterFS can also be used in Hadoop, in place of HDFS.

\section{Quantcast File System}

Quantcase File System can also be used in Hadoop, in place of HDFS.