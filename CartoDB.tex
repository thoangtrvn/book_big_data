\chapter{CartoDB - geospatial data}
\label{chap:CartoDB}

CartoDB was created to solve the problem with current tools when creating online
maps and dealing with geospatial data.
\begin{itemize}
  \item PostGIS: fastest geospatial database, based on PostgreSQL
  
  \item Mapnik:  

  \item MapBox: Sect.\ref{sec:MapBox}
    
  \item Fusion Tables: a  cell of data supports max 1 million characters; when
  zoomed farther out, tables with > 500 features will show dots (not lines or
  polygons), SQL api allows 5 requests/second; upto 5 Fusion Table layers to a
  map, one among them can be styled with upto 5 styling rules.
  
  \url{https://speakerdeck.com/jatorre/cartodb-how-working-with-geospatial-data-can-be-a-joy-and-not-a-pain}
\end{itemize}
250MB size limit, 500 vertices per tile limit, only the
  first 100K shapes will be rendered, only the first 100K rows of data in a
  table are mapped or included in the query results (after this, these rows are
  not displayed), max number of vertices supported per table is 5 million.
  

CartoDB is a new Open Source Geospatial Database in the cloud. 
CartoDB was built on open source software including PostGIS and PostgreSQL. The
tool uses JavaScript extensively in the front end web application, back end
Node.js based APIs, and for client libraries. 

It leverages the full power of PostGIS and Mapnik to effortlessly stand up the
services needed to create maps and develop location aware applications.
Importing geospatial data is as simple as dragging files into it, editing the
data is just the way it is supposed to be, the maps are fast and responsive, it
just works! Plus set the tables to real time and you will be able to have real
time visualizations on your data changes.



\section{CartoDB 1.0}

CartoDB 1.0 uses PostGIS 2.0 that supports
\begin{itemize}
  \item Raster
  \item Topology
  \item Nearest neighbors
  \item GeoJSON
  \item PostgreSQL 9.1
  \item ST\_MakeValid, ST\_Snap
  \item typmod
  \item OpenStreeMap
  \item GeoProcessing on JavaScript
\end{itemize}

\section{Carto3D = CartoDB + WhirlyGlobe}


\section{Vizzuality HTML5 Lab}

Contains example using CartoDB and HTML5: available on github
\url{}

\section{OpenStreeMap}


\section{MapBox}
\label{sec:MapBox}

