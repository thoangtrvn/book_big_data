\chapter{Tree Data Structures}
\label{chap:Tree_DataStructure}

\section{Introduction}

\subsection{Minimum fill in tree}


\subsection{BST}

Binary search trees (BST), sometimes called ordered or sorted binary trees, are
a class of data structures used to implement lookup tables and dynamic sets.

Binary search trees keep their keys in sorted order, so that lookup and other
operations can use the principle of binary search: when looking for a key in a
tree (or a place to insert a new key), they traverse the tree from root to leaf,
making comparisons to keys stored in the nodes of the tree and deciding

\url{http://en.wikipedia.org/wiki/Binary_search_tree}

There are many types of binary search trees:
\begin{enumerate}
  \item self-balancing binary search tree - Sect.\ref{sec:self_balancing-binary-search-tree}
  \item splay tree: 
\end{enumerate}

\subsection{self-balancing binary search tree)}
\label{sec:self_balancing-binary-search-tree}

It is a node-based binary search tree that automatically keeps its height
(maximal number of levels below the root) small in the face of arbitrary item
insertions and deletions. It is being used to implement {\it mutable ordered
lists}, or other abstract data structures (e.g. associative arrays, priority
queues, sets).

Two forms of self-balancing binary search tree
\begin{enumerate}
  \item Red-Black tree
  \item AVL tree :
\end{enumerate}

\section{Splay tree}
\label{sec:tree_splay}

A splay tree automatically moves frequently accessed elements nearer to the root
 
 
\section{B-tree (1972)}
\label{sec:tree_B}

The B-tree is a generalization of a binary search tree in that a node can have
more than two children.
Data is sorted and allows searches, sequential access, insertions, and deletions
in logarithmic time.
B-tree is optimized for systems that read and write large blocks of data, i.e.
widely used in database systems.

B-trees guarantee 50\% page fill

\section{B*-tree}

B*-trees guarantee 66\% page fill, higher than B-tree and R-tree.

\section{R-tree (1984)}
\label{sec:tree_R}


R-tree is widely used to store/index multi-dimentional information, e.g. spatial
objects. The spatial object is represented by a rectangle, i.e. the "R" in R-tree is for rectangle.

Example of spatial objects are restaurant locations, polygons (e.g. streets,
buildings, outlines of lakes, coastlines) that make of map.
It accelerates nearest-neighbor search for various distance metrics.

This allows the system to quickly find an element, such as ``find all museums
within 2 km of my current location'', or ``retrieve all road segments within 2km
of my current location'', or ``find the nearest gas station''. The key idea is
to use the bounding boxes to decide whether or not to search inside a subtree.

The key idea of the data structure is to group nearby objects and represent them
with their minimum bounding rectangle in the next higher level of the tree. So
the leaf is the isolated spatial objects, the parent node is the group of nearby
spatial objects. This parent node is also a spatial object, represented by a
minimum bounding rectangle.

Similar to the B-tree, the R-tree is also a balanced search tree (so all leaf nodes are at the same height).
It organizes the data in pages, and is designed for storage on disk.
Each page can contain a maximum number of entries, often denoted as M.

MINIMUM FILL: It also guarantees a minimum fill (except for the root node).
However, best performance has been experienced with a minimum fill of 30\%-40\%
of the maximum number of entries. The reason for this lower minimum fill is the
more complex balancing required for spatial data as opposed to linear data
stored in B-trees.

The key difficulty of R-trees is to build an efficient tree that on one hand is
balanced (so the leaf nodes are at the same height) on the other hand the
rectangles do not cover too much empty space and do not overlap too much (so
that during search, fewer subtrees need to be processed).

\section{R*-tree (1990)}

R*-trees have slightly higher construction cost than standard R-trees, as the data may need to be reinserted; but the resulting tree will usually have a better query performance.
Like the standard R-tree, it can store both point and spatial data. 




\section{S-tree}
\label{sec:tree_S}



\section{P-tree}
\label{sec:tree_P}


\section{SP-tree}
\label{sec:tree_SP}



\section{Nearest-neighbor searchs}


We can store data in
\begin{itemize}
  \item R-tree (Sect.\ref{sec:tree_R})
\end{itemize}


\section{Fanout tree (chip design) - 1989}
\label{sec:tree_fanout}

In a chip, some gates have to drive many sinks or long wires. 
The propagation delay of the signal on these gates can be improved by repowering the signal with little amplifiers, called {\bf buffers}.

To come up with a best distribution of the buffers, to optimally distribute the
signals over the chip, a tree structure called {\bf fanout tree} is used.
 
Berman et al. (1989) 
\url{http://dl.acm.org/citation.cfm?id=90921}


\section{RC-tree (chip design) or van Ginneken algorithm - 1990}
\label{sec:tree_RC}

The RC-tree is a better version of fanout tree (Sect.\ref{sec:tree_fanout}), as it take into account the physical information such as capacitance and RC effects of the wires.

The practical issue is that a large portion of the delay in an IC (integrated circuit) is due to the time it takes to charge and discharge the capacitance
of the wires and the gates of transistors. Both resistance R and capacitance C increase linearly with the length $l$
\begin{equation}
R \approx   r.l \\
C \approx c.l
\end{equation}

The RC delay of the wire is 
\begin{equation}
D \approx 1/2RC = 1/2 r.c.l^2
\end{equation}
which increases quadratically with the length of the wire. This shows that buffering the wiring trees after placement or floor planning is very important.

The RC-tree algorithm developed by van Ginneken is based on 
\begin{itemize}
  \item the root of the tree is the source of the signal
  \item the leafs are the sinks of the signal
  \item RC-tree model a tree of distributed RC sections, i.e. a node has a resistor R and a capacitance C.
  
  \item the capacitance of the surrounding wires is modeled as an extra contribution to the capacitance to ground.
  \item as computing the exact delay of a wiring tree is difficult, i.e. requires the solution of a set of differential equaitons for the distributed RC sections, a simple estimation of the delay called {\bf Elmore delay} is used.
  
  \url{http://en.wikipedia.org/wiki/Elmore_delay}
  
  \item other assumptions: preserve charges (i.e. no leakage to ground)  
\end{itemize}
Elmore delay is the objective function of the algorithm, i.e. minimize Elmore delay. The Elmore delay is defined as the first order moment of the impulse response $h(t)$
\begin{equation}
D = \int^\infty_0 h(t)dt
\end{equation}




van Ginneken (1990) \url{http://xilinx.asia/_hdl/4/eda.ee.ucla.edu/EE201A-04Spring/buffer_insertion.pdf}
\url{http://eda.ee.ucla.edu/EE201A-04Spring/lecture9_buffering.ppt}


\section{K-d tree: spatially aware data structure}

\url{https://www.afsbirsttr.com/award/AwardDetails.aspx?pk=20392}

\section{Spatial Hash Table: spatially aware data structure}

\url{https://www.afsbirsttr.com/award/AwardDetails.aspx?pk=20392}