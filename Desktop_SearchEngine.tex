\chapter{Desktop Search Engines}
\label{chap:Desktop_SearchEngine}

\url{http://en.wikipedia.org/wiki/List_of_search_engines\#Desktop_search_engines}

\section{Google Desktop}
\label{sec:Google_Desktop}

Google Desktop is a discontinued product, due to the huge shift from local to
cloud-based storage and computing, as well as the integration of search and
gadget functionality into the modern operating system (O/S).
\begin{enumerate}
  \item Email messages
  \item Computer files
  \item Music
  \item Photos
  \item Chats
  \item Web pages viewed
  \item Google gadget on user's desktop sidebar of Windows
\end{enumerate}

To display on Linux's desktop sidebar, use
\url{https://code.google.com/p/google-gadgets-for-linux/}

\section{DocFetcher}
\label{sec:DocFetcher}

DocFetcher is open-source multi-platform, written in Java with standard Widget
Toolkit for GUI. The searching and indexing capabilities of DocFetcher is based
on Apache Lucene (Sect.\ref{sec:Lucene}).

It can parse text from documents in
\begin{enumerate}
  \item formats: PDF, HTML, EPUB, MS Office, OpenOffice
  \item zip file: zip, 7z, rar, tar.*
  \item Outlook emails: PST files
  \item customized to search in any kind of source-code files
\end{enumerate}
It also 
\begin{itemize}
  \item automatically updates the indexes (when files are modified)
  \item feature to exclude files from being indexed using regular expressions
\end{itemize}


\url{http://en.wikipedia.org/wiki/DocFetcher}

\section{Find and Run Robot}