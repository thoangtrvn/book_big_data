\chapter{Apache Hive and Pig}
\label{chap:Hive_PrestoDB}

The very earliest versions of Hadoop supported exactly one way of getting at
data: you need to write your Java code with MapReduce. To enable non-programmers
to work with the data, Yahoo developed a language of its own, called {\bf
Apache Pig} (Sect.\ref{sec:pig}). Facebook, on the other hand, developed
SQL-system called {\bf Apache Hive} (Sect.\ref{sec:hive}) rather than training
people with a new query language.

{\bf LIMITATIONS}: Like Hadoop's MapReduce, Pig and Hive inherit the
batch-processing property. To provide real-time processing, Impala was
developed 

\section{Hive}
\label{sec:hive}

Apache Hive provides a facilities for query data stored on HDFS via an SQL-like
language, called HQL. The queries are translated into MapReduce jobs. 
ALso, data query in Apache Hive is not real-time, but is batch-job processing.

The reason for developing Apache Hive is that it allows SQL savvy to be able to
use data on HDFS easily. As Apache Hive is JDBC compliant, existing SQL-based
tools can integrate with Apache Hive to extract data stored in HDFS. 
Running Hive queries could take a while since they go over all of the data in
the table by default.

{\bf LIMITATIONS:} \textcolor{red}{Hive does not currently support update
statements}. Additionally, since it runs batch processing on Hadoop, it can take
minutes or even hours to get back results for queries. Hive must also be
provided with a predefined schema to map files and directories into columns and
it is not ACID compliant.

{\bf USAGE:} Hive should be used for analytical querying of data collected over
a period of time. 

{\bf NOT TO USE HIVE:} Hive should not be used for real-time querying since it
could take a while before any results are returned. In that case, we should use
HBase. 


\section{Pig}
\label{sec:pig}

Rather than provide SQL-like language, Apache Pig provides scripting language
for accessing and transforming data.
