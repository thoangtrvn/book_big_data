\chapter{Cloud Computing}
\label{chap:cloud_computing}


In 2006, Amazon launched their Amazon Web Services (AWS) public cloud platform,
and a new era began. The era of cloud-native applications (Sect.\ref{chap:cloud-native-apps}).

Cloud computing has emerged as an interesting variant on enterprise computing
(Sect.\ref{sec:enterprise-computing}).
It provides the capability of flexible allocation of compute resources
(including processing, network and storage) to support an ever-shifting pool of
applications, i.e. different business need different softwares that can be
quickly deployed to the cloud.

{\it Cloud computing is a computing model}, where resources such as computing
power, storage, network and software are abstracted and provided as services on
the Internet in a remotely accessible fashion.
The Cloud refers to the capability of using some service (software and/or
hardware), provided and hosted somewhere by a cloud-computing provider company,
from your local computer without having to install the code (e.g. program) or
infrastructure performing that service. So, this service is not necessary a
soft-part but can also be a hard-part (e.g. Infrastructure).

This leads to many Cloud Service acronyms such as IaaS, PaaS,
SaaS, Maas, Caas or Xaas.
\begin{itemize}
  \item {\bf IaaS} (Infrastructure as-a-Service) or {\bf HaaS} (Hardware
  as-a-Service):   the consumer use the infrastructure outsourced by the service
  provider.
  
   The Service Provider not only owns the equipment but will also be responsible
  for its running and maintenance, where the consumer will be charged on a 'pay
  as you use' basis. IaaS is often offered as a horizontally integrated service
  that includes not only the server and storage but also the connectivity
  domains. This often goes with PaaS, where the consumer select the scope of the
  service to use (e.g. the environment (O/S, libraries, memory, storage
  capcity), and how many machines) The Iaas provider would typically provide the
  replication, backup and archiving (Storage), the powerful computing
  requirements (Server) or the network load balancing and firewalls
  (Connectivity domains).
  
\begin{mdframed}
IaaS gives you components you need in order to build things on top of it; PaaS
gives you an environment where you just push code and some basic configuration
and get a running application. IaaS gives you more power and flexibility, at the
cost of having to build more yourself.    
\end{mdframed}  
  
  \item  {\bf PaaS} (Platform as-a-Service): The service provider has a web
  application that enables the consumer to have a control over the deployed
  environment, without the complexity of the infrastructure
  (Sect.\ref{sec:PaaS}).
  
  PaaS facilitates immediate business requirements such as application design,
  development and testing at a fraction of the normal cost.
  
  \label{sec:SaaS}
  \item SaaS (Software as-a-Service): a consumer to use on demand software that
  is provided by the service provider via a thin client device, e.g. a web
  browser over the Internet. This include PaaS and IaaS. With SaaS, the consumer
  has not only no management or control of the infrastructure such as the
  storage, servers, network, or operating systems, but also no control over the
  application's capabilities.
  
  This include Google Docs, Google Calendar.
  SaaS is a quick and efficient delivery model for key business applications
   such as customer relationship management (CRM), enterprise resource planning
   (ERP), HR and payroll. 
  
  \item {\bf MaaS} (Monitoring as-a-Service): at present still an emerging piece
  of the Cloud jigsaw but an integral one for the future. 
  
  \item {\bf Maas} (Metal as-a-Service): a provisioning construct created by
  Canonical, to help facilitate and automate the deployment of hyperscale
  computing environment such as big data workloads and cloud services. 
  
  Maas serves as a layer underneath IaaS and works with Joju to coordinate
  applications and workloads. (Sect.\ref{sec:Maas_Ubuntu})
  
  
  \item {\bf CaaS} (Communication as-a-Service):  enables the consumer to
  utilize Enterprise level VoIP, VPNs, PBX and Unified Communications without
  the costly investment of purchasing, hosting and managing the infrastructure.
  
  
  With the service provider responsible for the management and running of these
  services also, the other advantage the consumer has is that they needn't
  require their own trained personnel, bringing significant OPEX as well as CAPEX costs. 
  
  \item {\bf DBaas} (Database as-a-Service): the user can put their data on a
  remote database, hosted by some cloud platform, without using any virtual machine
  instance for the database. In this configuration, application owners do not
  have to install and maintain the database on their own. 
  
  
  \item {\bf XaaS} (anything as-a-Service):  
  
\end{itemize}
 \url{http://www.zdnet.com/article/cloudy-concepts-iaas-paas-saas-maas-caas-xaas/}
 
With cloud computing service, you have the option to choose a computing cluster
to rent. There are different companies ofering cloud computing services 
\begin{itemize}
  \item Google cloud platform (Sect.\ref{sec:google_cloud-platform}) - platform
  as a service
  \item Google Apps (Sect.\ref{sec:google_apps-work}) - software as a service
  
  \item Amazon EC2 platform
\end{itemize}
You choose the one you like (size, capacity, environment configuration) without
having to spend the money on buying infrastructure, maintaining the system. You
pay for what you need. This is the growing trend for small or mid-size companies
which do not want to maintain a good technical team, and also good for prototype
testing.

However, a company can also build their own cloud, using {\bf OpenStack} or
Eucalyptus (Sect.\ref{sec:Eucalyptus}).

\section{Enterprise computing}
\label{sec:enterprise-computing}	


Enterprise computing is a buzzword that refers to business-oriented information
technology that is critical to a company's operations. This typically requires
using high-ended computing facilities, and various types of enterprise
softwares.

In recent years, cloud computing is emerging as a good option for business, to
reduce the upfront cost, and can be scaled if needed easily.

\section{Cloud Operating System}

A hypervisor or virtual machine manager (Sect.\ref{sec:hypervisor}) 
is a special software that allows you to run different guest O/S on host O/S.
The guest O/S is saved in the formed of a virtual machine instance and this
VM instance can be easily be deployed on any machine. This is the core from
which cloud computing is created.

A {\bf Cloud O/S} (or known as {\bf Intra-VM O/S}) is an O/S that is supposed to
run on a {\bf container} (Sect.\ref{sec:containers_Linux}), rather than
hypervisor. Also, this cloud O/S running on containers is different from the
traditional O/S running on hypervisors.
\begin{itemize}
  \item  The traditional O/S were designed to be run on hardware,
  so they have all the complexity needed for a variety of hardware drivers from
  an assortment of vendors with different design concepts. These operating
  systems are also intended to be multi-user, multi-process, and multi-purpose.
  They are designed to be everything for everyone, so they are necessarily
  complex and large.

  \item The cloud O/S are designed to be single-user purpose. Also, it is not
  designed to run on hardware, and so lacks the bloat and complexity of driver.
  It is not meant to be multi-user or multi-process, so it can focus on creating
  a single thread of code which runs one application, and one application only.
  
  Most are not multi-purpose, as the target is to create a single payload that a
  particular instance will execute (OSv is an exception)
  
\end{itemize}
\url{http://wiki.xenproject.org/wiki/Cloud_Operating_Systems#Cloud_Operating_Systems_Versus_Linux_Containers}


IN SUMMARY: Generally, a cloud O/S generate an environment that lacks the
ability to spawn subprocesses, execute shell commands, create multiple threads,
or fork processes. Instead, they provide a pure incarnation of the language
runtime targetted, be it OCaml, Haskell, Java, Erlang, or some other
environment.
The Cloud O/S are thus small, lightweight, and quick. These Cloud Operating
Systems may become the core of a new form of cloud, where a single hypervisor
instance, known as {\bf containers} (Sect.\ref{sec:containers_Linux}) can
support hundreds or even thousands of VMs.
 
% improves IT
% resource utilization by treating your company's physical resources as pools from which virtual resources can be dynamically allocated. Using virtualization in your environment, you are
% able to consolidate resources such as processors, storage, and networks into a
% virtual environment which provides the following benefits:
% \begin{itemize}
%   \item reduce hardware cost
%   \item optimize workloads
%   \item IT flexibility and responsiveness
% \end{itemize}

% To help manage the operation, execution and processes of VMs, Virtual Servers
% and Virtual Infrastructure as well as the back-end hardware and software
% resources, we need a cloud operating system or a {\bf virtual operating system}.
% It is a virtual operating system because it needs to run on top of another
% operating system, e.g. Linux.

The four different purported meanings of "cloud operating system", though the
last one Intra-VM O/S is the one people think of first

\begin{enumerate}
  \item A browser-based desktops: a machine with a simple O/S which functions
  like a virtual desktop, i.e. central storage of data and a basic set of
  typical office applications.
  
  Example: JoliOS, Glide OS, SliveOS, ZeroPS, Cloudo, Google Chrome OS.
  
  \item Marketing hype: just something that a company (Microsoft Azure, HP,
  Oracle, \ldots) used to bill their customers a service that they call a cloud
  operating system.
  
  
  \item Intra-VM O/S (Cloud O/S): a lightweight O/S that is tailored and
  customized to run more efficiently as a virtual machine on a {\bf container}
  (Sect.\ref{sec:containers_Linux}), which in turn can run on a public, private
  or hybrid cloud environment.
  
  Example: OSv (produce Java environment), Mirage O/S (use
  OCaml-based kernel), LING (create Erlang environment), HaLVM (Haskell
  environment), ClickOS (high-performance uni-kernel producing network devices)

  \item Roll-your-own cloud software: it is a special type of software that
  enable you to create a cloud computing environment, primarily a private cloud
  inside an organization. Here, the software manage a set of networked machies
  to deliver cloud services make sense. 
  
  Example: OpenStack
  
  
\end{enumerate}
Based on Michael C. Daconta, only OpenStack deserves the name cloud O/S as it
provides the full operating system each run on a physical machine
\footnote{\url{http://gcn.com/Blogs/Reality-Check/2014/01/cloud-operating-system.aspx}}

% Example of cloud operating system
% \begin{enumerate}
%   \item OSv
%   \item Windows Azure
%   \item Google Chrome OS
%   \item OpenStack
% \end{enumerate}

A cloud operating system is a type of operating system designed to operate
within cloud computing and virtualization environments. System virtualization
creates many virtual systems within a single physical system. Here, we need a
Cloud Operating System or Cloud Platform. 

A Cloud O/S provides many layers of abstraction.


\subsection{OSv (customized O/S to run as a guest VM)}
\label{sec:OSv}

A group of senior programmers who created KVM (Sect.\ref{sec:KVM}) created OSv -
a customized operating system for virtual machine. This cloud operating system
is designed to run applications in virtual machines extremely fast via a
stripped down version of Linux. The applications must be either Java, Ruby or
POSIX applications. For this reason, it does not support a notion of users (it's not a
multiuser system) or processes - everything runs in the kernel address space.

OSv can be controlled by hypervisor like KVM (Sect.\ref{sec:KVM}), Xen
(Sect.\ref{sec:Xen}), VMWare's ESXi (Sect.\ref{sec:VMWare_ESXi}), or VirtualBox
(Sect.\ref{sec:VirtualBox}). OSv bundles an app, application server, and Java
Virtual Machine into a hypervisor that sits on top of hardware - no Linux OS
needed. It use ZFS filesystem (Sect.\ref{sec:filesystem_block-oriented}) and
claimed to have sub-1 second bootime (i.e. the boot time is less than 1 second)
\footnote{\url{http://www.embedded-bits.co.uk/2010/1-second-linux-boot-time/}},
while matching or outperforming Linux guests on VMs on benchmarks around
SpecJVM, MemCacheD, Cassandra, and TCP/IP.

OSv lacks support for multiple address spaces - which boots its performance -
and instead of mandating separation between applications, the kernel uses the
JVM to block accesses to kernel memory. These two methods free up resources for app performance. 
Also, low level OS mechanisms such as memory management, scheduling and IO are
accessible for the JVM directly, resulting in unparalleled throughput and
latency.

Since the JVM protects itself from the application by verifying bytecode, OSv
does not need to do this. The application and the kernel run in the same
privilege level, and so expensive context switches and parameter validation are
avoided.

Ideal workloads for the system include Java application servers like Tomcat, or
C-based applications ported by OSv developers, such as MemCacheD, redis,
nginx, or MongoDB. 

\url{http://en.wikipedia.org/wiki/OSv}

\url{http://www.slideshare.net/rhatr/osv-probably-the-best-os-for-cloud-workloads-youve-never-hear-of}

\url{http://www.theregister.co.uk/Print/2013/09/17/cloudius_systems_osv_cloud_software/}



\subsection{OpenStack (IaaS) - help managed networked machines as a cloud
environment}

OpenStack is a control layer that sits above all the virtualized layers and
provides a consistent way to access everything regardless of the hypervisor
technology (Sect.\ref{sec:hypervisor}) used (e.g.: KVM, Xen, vmware, etc.)
underneath.
\url{http://getcloudify.org/2014/07/10/what-is-openstack-tutorial.html}

OpenStack provides IaaS solution through a variety of complemental services.

OpenStack is a cloud operating system that controls large pools of compute,
storage, and networking resources throughout a data center. All of the above
components are managed through a dashboard which gives administrators control



For information on how to deploy OpenStack - Chap.\ref{chap:OpenStack}.

 
\section{PaaS (Cloud Platform services)}
\label{sec:PaaS}

Many company uses cloud computing platform to provide 'renting' service, e.g.
users pay by the hours for active server instances; hence the term {\bf
elastic}. Additional features can be provided
\begin{itemize}
  \item users should be able to select the geographical location of the server
  instances (to minimize latency) and choose different level of redundancy.
\end{itemize}

The idea of PaaS was pioneered by Amazon Web Service (AWS -
Sect.\ref{sec:Amazon_AWS}) and Salesforce.com. In 2008, Google launched App
Engine (Sect.\ref{sec:Google_AppEngine}).

Originally, all PaaS were in the public cloud, then some companies want to
deploy their own private and hybrid PaaS options (i.e. managed by their own IT
departments).

Public PaaS is situated in cloud computing in between Software as-a-Service
(SaaS) and Infrastructure as-a-Service (IaaS). SaaS is software that is hosted
on the cloud; while IaaS provides virtual storage from a provider with
adjustable scalability. With IaaS, the users have to manage the server; while
with PaaS, the server management is done by the provider.

Private PaaS is a software that can be downloaded and installed on a company's
on-premises infrastructure, or in a public cloud. Once it is installed on one or
many machines, it arranges the applications and databases components into a
single hosting platforms. Example:
\begin{enumerate}
  \item Apprenda: provide .NET support
  \item Red Hat's OpenShift
  \item Pivotal Cloud Foundry
\end{enumerate}

Mobile PaaS (mPaaS):
\begin{enumerate}
  \item Kinvey
  \item CloudMine
  \item AnyPresence
  \item FeedHenry
  \item FatFractal
  \item Point.io
\end{enumerate}


 (e.g. Heroku -
Sect.\ref{sec:Heroku})

% Example:
% \begin{itemize}
%   \item Heroku
%   \item 
% \end{itemize}

\subsection{Heroku}
\label{sec:Heroku}

Heroku is PaaS; while Amazon AWS (Sect.\ref{sec:Amazon_AWS}) is IaaS.
Recently, AWS does actually have a PaaS offering, Elastic Beanstalk, that
supports Ruby, Node.js, PHP, Python, .NET and Java.

To get your code running on AWS and looking a bit like a Heroku deployment,
you'll want some EC2 instances - you'll want a load balancer / caching layer
installed on them (e.g. Varnish), you'll want instances running something like
Passenger and nginx to serve your code, you'll want to deploy and configure a
clustered database instance of something like PostgreSQL. You'll want a
deployment system with something like Capistrano, and something doing log
aggregation.      

\url{http://stackoverflow.com/questions/9802259/why-do-people-use-heroku-when-aws-is-present-whats-distinguishing-about-heroku}

Heroku is one of the first cloud platforms (using Debian as the base O/S, and currently is using
Ubuntu), developed since 2007, with language supports
\begin{itemize}
  \item Ruby (the first one) and support Rack-based objects (web server interface) for 
  writing web applications.
  
  \item Java
  \item Node.js
  \item Scala
  \item Clojure
  \item Python
  \item PHP
  \item Perl
\end{itemize}
and database
\begin{itemize}
  \item PostgreSQL
  \item MongoDB
  \item Couchbase server
  \item Cloudant
  \item Redis
\end{itemize}

The app running on Heroku has the domain name: applicationname.herokuapp.com
and resolved by Heroku DNS server.

Heroku is now part of Salesforce.com.


Writing apps running on Heroku platform:
\begin{itemize}
  \item scheduler feature: \url{https://addons.heroku.com/scheduler}
  simple scheduling (regular 10 mins, hourly, daily interval)
  
  \item APScheduler: more complicated scheduling (every 5min, or 37mins or
  at specific time)
\end{itemize}

\section{Cloud services: Amazon AWS}
\label{sec:Amazon_AWS}

Amazon Web Services (AWS) provides an option to select any of the 18 products
and services
\begin{enumerate}
  \item Amazon EC2: provide an EC2 instances (e.g. Windows or Linux ) which
  works as remote computer, on which we can install whatever software you want,
  including a web server running PHP code and a database server. 
  
  We can scale up or down easily the number of instances we need. It charges by
  the amount of hours you run the instances.
  
  Existing instances: 
  \begin{itemize}
    \item Linux, RHEL, SLES t2.micro instance
    \item Windows t2.micro instance
  \end{itemize}

  EBS is the file system of the EC2 instance itself, kinda like NTFS or ext4.
  AmazonS3 can be seen as an external storage device with high capacity and high
  availability

  \item Amazon EBS: persistent data storage that can attach to an EC2 instance

Amazon Elastic Block Storage (EBS) provides raw block device that can be
  attached to Amazon EC2 instances, and can be used like any raw block device,
  e.g. formatting the devices with a given filesystem
  (Sect.\ref{sec:file_system}) and mounting it. Replication is supported to
  avoid data loss due to failure of a single component.
  
  EBS volume can be:  upto 1TB (June 2014)
  
  \url{http://en.wikipedia.org/wiki/Amazon_Elastic_Block_Store}
  
  \item Amazon S3 (Simple Storage Service): provide a persistent storage
  service, typically to store large binary files. In additional to binary data,
  you can choose services that provide No-SQL database server (e.g. AWS
  DynamoDB) or SQL database servers (e.g.
  Amazon RDS).
  
  It charges by the amount of storage and the number of data access
  Default: 5GB and 20,000 Get requests + 20,000 Put requests.
  
  S3 was designed for content storage. The correct Amazon service to use for
  content delivery is Amazon CloudFront. 
  
  \item Amazon DynamoDB: provide No-SQL database server
  (Sect.\ref{sec:DynamoDB})
  
  \item Amazon RDS: provides option to choose MySQL, Postgres, Oracle, SQL
  Server, and Amazon Aurora.

Database service is provided via Amazon RDS (Relational Database Service) web
service, and it uses Oracle, SQL Server, MySQL or PostgreSQL, with encryption
option. Recently, RDS supports Aurora, a MysQL-compatible, relational database
engine (Sect.\ref{sec:Aurora}).
\url{http://aws.amazon.com/rds/}.

  \item AWS Lambda: a compute service (since Nov-2014), run code without
  provisioning or managing servers by creating a Lambda function, and then
  connect the function to your AWS resources)
  
  Lambda will automatically run code in response to modifications to objects
  uploaded to Amazon S3 buckets, messages arriving in
  Amazon Kinesis streams, or table updates in Amazon DynamoDB.
  \url{https://aws.amazon.com/blogs/aws/run-code-cloud/}
  
  \item AWS CloudFront: 
  
  
\end{enumerate}

Free account for 12 months: AWS Free Tier includes 30GB of Storage, 2 million
I/Os, and 1GB of snapshot storage with Amazon Elastic Block Store (EBS).
\begin{itemize}
  \item EC2: 750 hrs/month of Windows or Linux t2.micro instance usage
  \item S3: 5GB storage
  \item RDS: 750 hrs/month of Micro DB instance usage
  \item DynamoDB: 25GB storage, upto 200 million requests/month
  
  DynamoDB differs from other Amazon services by allowing developers to purchase
  a service based on throughput (i.e. how many requests per month), rather than
  storage.
\end{itemize}

\subsection{EC2}


Amazon EC2 supports 2 platforms (check your account to see which one is
supported)
\url{https://docs.aws.amazon.com/AWSEC2/latest/UserGuide/ec2-supported-platforms.html}

\begin{enumerate}
  \item EC2-VPC: instances run a virtual private cloud (VPC) that is logically
  isolated to your account
  
   Each instance a private IP address from the private IP address range of your
  VPC. You can control the IP address range, subnets, routing, network gateways,
  network ACLs, and security groups for your VPC. You can specify whether your
  instance receives a public IP address during launch.
  
  Instances with public IP addresses or Elastic IP addresses can access the
  Internet through a logical Internet gateway attached to the AWS network edge
  
  \item EC2-Classic: instances run a single, flat network that is shared with
  other customers
  
  Each instance is assigned a private IP address from a shared private IP
  address range; and a public IP address from Amazon's pool of public IP
  addresses. Instances access the Internet directly through the AWS network
  edge.
    
\end{enumerate}

EC2 instances are called Amazon Machine Image (AMI) which use one of two types
of virtualization: paravirtual (PV) or hardward virtual machine (HVM). For best
performance, use HVM.
\begin{itemize}
  \item Paravirtual AMIs boot with a special boot loader called PV-GRUB,
  \item HVM AMIs are presented with a fully virtualized set of hardware and boot
  by executing the master boot record of the root block device of your image. 
  
  Unlike PV guests, HVM guests can take advantage of hardware extensions that
  provide fast access to the underlying hardware on the host system.  This
  virtualization type provides the ability to run an operating system directly
  on top of a virtual machine without any modification, as if it were run on the
  bare-metal hardware.
  
\end{itemize}
\url{http://docs.aws.amazon.com/AWSEC2/latest/UserGuide/virtualization_types.html}

EC2 instances are all 64-bit and can be
\begin{itemize} 
  \item Linux AMI 2014.09.2 (HVM): an EBS-backed image. 
  \item Linux AMI 2014.09.2 (PV)
  
  The default image includes  command line tools, Python, Ruby, Perl, and Java.
  The repositories include Apache HTTPD, Docker, PHP, MySQL, PostgreSQL, and other packages.
  
  \item Red Hat Enterprise Linux 7.0, SSD volume type: EBS is SSD volume type
  \item Red Hat Enterprise Linux 6.5
  
  \item SUSE Linux Enterprise Server 12, SSD volume type:
  
  \item Ubuntu Server 14.04 LTS, SSD volume type
  
  \item Windows Server 2012 R2 base
  \item Windows Server 2012 R2 with SQL Server web
  \item Windows Server 2012 R2 with SQL Server standard
  \item Windows Server 2012 base
  \item Windows Server 2012 with SQL Server Express
  \item Windows Server 2012 with SQL Server Web
  \item Windows Server 2012 with SQL Server Standard
  
  \item Windows Server 2008 base
  \item Windows Server 2008 R2 base
  \item Windows Server 2008 R2 with SQL Server Express and IIS
  \item Windows Server 2008 R2 with SQL Server Web
  \item Windows Server 2008 R2 with SQL Server Standard
\end{itemize}

Factors to consider an Amazon AMI
\url{http://docs.aws.amazon.com/AWSEC2/latest/UserGuide/finding-an-ami.html}

After you choose the Amazon AMI instance, your next step is to choose the host
machine to deploy. The choice depends on the purpose (1) storage optimized
(i.e. HI1 instance for No-SQL or HS1 instance for large-scale datawarehouse),
(2) memory optimized (for database app, memcached or distributed caches apps
such as SAP or Microsoft SharePoint)

\begin{itemize}
  \item \verb!t2.micro!
  \item \verb!t2.small!
  \item \verb!t2.medium!
  \item \verb!m3.medium!
  \item \verb!m3.large!
  \item \ldots
  \item \verb!c3.8xlarge! 
  \item \verb!g2.2xlarge!: backed by one Kepler GK104 GPGPU and 8x hyperthreads
  Intel Xeon E5-2670
\end{itemize}

One EC2 Compute Unit provides the equivalent CPU capacity of a 1.0-1.2 GHz 2007
Opteron or 2007 Xeon processor.


\subsection{EBS}

% \subsection{Amazon}
% 
% Amazone EC2 (Elastic Compute Cloud), and AWS (Amazon Web Service).
% The AWS allows user to create the server instances and then choose the
% configurations and the price plan to use the Amazon EC2. Other options for data storage:
% on Amazon Storage (S3) that uses EBS; or on noSQL-database
% DynamoDB .


\section{Cloud services: Google}

\begin{enumerate}
  \item Google Cloud Storage: 
  
  \item Google App Engine: 
\end{enumerate}


\section{Cloud Management Platform} 

A Cloud Computing Platform is a web-based platform that enables users to 
\begin{enumerate}
  \item create a virtual computer (i.e. a server instance), via an image (e.g.
  Amazon machine image)
  
  The image should contain any software desired
  
  \item create, launch, terminate server instances as needed
\end{enumerate}


Chap.\ref{chap:multi-cloud-management} 

\subsection{Ubuntu}

Want to minimise the time to get your cloud infrastructure up and running and
start offering your services to customers? 


\url{http://www.ubuntu.com/cloud/tools}
 
\section{Google}

\subsection{Google: Google Apps for Work}
\label{sec:google_apps-work}

Google Apps for Work (formerly: Google Apps) is a cloud computing
solution that provide free web applications (Gmail, Google Docs, Google
Calender, Google Drive), and business-specific features (e.g. ) and runs on
Google Cloud Platform (Sect.\ref{sec:google_cloud-platform})

With web applications, the data is hosted by Google's network of secure data
centers.



\subsection{Google: Google Cloud Platform}
\label{sec:google_cloud-platform}

Google Cloud Platform is a cloud computing platform hosted on the infrastructure
provided by Google, which is the same as the infrastructures being used
internally by Google to host Google Search and Youtube. 

By using this cloud platform, users can develop a range of applications from
simple website to complex applications. 

Google Cloud Platform provides
\begin{itemize}
  \item storage: regular file (Cloud Storage), database (Cloud SQL), Cloud
  Datastore
  
  \item running environment: Compute Engine, App Engine (Chap.\ref{chap:AppEngine})
  
  Google Compute Engine is an IaaS product offering flexible, self-managed
  virtual machines hosted on Google. It can be Linux-based VM running on KVM,
  local and durable storage options, can be configured and control via a simple
  REST based API.
  
  Google App Engine allows users to control Google Compute Engine cores and
  offers a web facing front end for Google Compute Engine data processing
  applications. \url{https://cloud.google.com/compute/docs/faq}
  
  \item application services: BigQuery, Cloud Endpoints
\end{itemize}

\subsection{Google: Google search for work}

\subsection{Google: Google Map for work}
 


\section{Ubuntu for Cloud Computing}

Ubuntu Enterprise Cloud (UEC) was introduced to Ubuntu 10.04.1 (Lucid) using
Eucalyptus (Sect.\ref{sec:Eucalyptus}) that runs on hypervisors like KVM, Xen
and VMWare. 
\url{https://help.ubuntu.com/community/UEC}

Since Ubuntu 11.0, UEC is replaced by Ubuntu Cloud Infrastructure (UCI) which is
based on OpenStack.
\url{https://help.ubuntu.com/community/UbuntuCloudInfrastructure11.10}
\begin{itemize}
  \item Install Orchestra server
\begin{verbatim}
sudo apt-get update
sudo apt-get install ubuntu-orchestra-server -y
\end{verbatim}  
  
  \item Install Juju (on the same server where Orchestra is installed or a
  different one)
  
  \item Deploy UCI with Juju
\end{itemize}
REQUIREMENTS:
\begin{itemize}
  \item  all nodes with two network interfaces linked to two seperate physical
  networks. 
  
  Deploying with a single network is possible with some hacking and workarounds,
  see original document for that. 
  
  \item Minimum 6 nodes are required, including a node for juju bootstrap node.
\end{itemize}

There is several changes in UCI since Ubuntu 12.04 LTS that use OpenStack
Essex release 
\url{https://help.ubuntu.com/community/UbuntuCloudInfrastructure}



 \section{MAAS Ubuntu}
 \label{sec:Maas_Ubuntu}

 
MAAS as first introduced into Ubuntu 12.04 LTS. It is a provisioning tool by
Canonical for bridging cloud semantics to the bare metal world using Ubuntu
O/S. \url{http://www.webopedia.com/TERM/M/metal-as-a-service_maas.html}

For more information, read Sect.\ref{sec:MAAS}.

\section{Cloud storage - security}

One thing that stops a lot of people from using cloud storage companies is the
perceived lack of security. 

\subsection{BoxCryptor (commercial)}

BoxCryptor's encryption helps fill that need. It is based on EncFS.
\footnote{\url{http://en.wikipedia.org/wiki/EncFS}} BoxCryptor is a free Windows
desktop app that creates an encrypted folder that can be placed inside your cloud storage folder.

Once you create the folder and assign a password, simply drag and drop the files
you want protected into that folder. BoxCryptor instantly encrypts and protects
them using the AES-256 standard. 

To unlock the folder and view your files, simply run BoxCryptor, navigate to the
encrypted folder and enter your password. If someone attempts to open the files without the password, an error message
will show. However, only the file contents are hidden: The file name and file
format are still in plain sight. So for super-duper extra security, change the
file name to something innocuous.

There is no master password retrieval process, i.e. \textcolor{red}{once you set
the password for the encrypted folder, if you forget the password, the files
inside the folder are lost forever.}
  
  
\subsection{TrueCrypt (stopped in 5/2014)}

TrueCrypt provides on-the-fly encryption (OTFE).
TrueCrypt supports Windows, OS X and Linux operating systems, both 32-bit and
64-bit, except Windows IA-64 and Mac OS 10.6 32-bit. The version for Windows 7,
Windows Vista, and Windows XP can encrypt the boot partition or entire boot
drive.

TrueCrypt supports  AES, Serpent, and Twofish, along with five different
combinations of cascaded algorithms are available: AES-Twofish,
AES-Twofish-Serpent, Serpent-AES, Serpent-Twofish-AES and Twofish-Serpent. 

TrueCrypt License is not considered "free" by several major Linux distributions
and is therefore not included in Debian, Ubuntu, Fedora,
openSUSE, or Gentoo. 

\url{http://en.wikipedia.org/wiki/TrueCrypt}


\url{http://truecrypt.sourceforge.net/}

\subsection{BitLocker}

BitLocker provides full disk encryption on Windows platform (for Windows Vista,
windows 7, 8, 8.1, Windows Server 2008). 

\subsection{EncFS}
\label{sec:EncFS}

\url{https://github.com/vgough/encfs}

\section{Big Data on the Cloud }
\label{sec:BigData-on-Cloud}

\subsection{OpenStack + Hadoop}
\label{sec:OpenStack+Hadoop}

OpenStack 2014.2 codedname Juno enable running Hadoop applications (including
Apache Spark - Sect.\ref{sec:apache_spark}) on OpenStack via the new feature:
Sahara.
\url{https://wiki.openstack.org/wiki/Sahara}

\url{http://www.zdnet.com/article/openstack-hooks-up-with-hadoop-to-bring-big-data-to-the-cloud/}

\chapter{Multi-cloud management}
\label{chap:multi-cloud-management}


So far, the public cloud market has seen increasing consolidation under the “big
four” suppliers— namely Google, Amazon, Alibaba, and Microsoft— which could
account for 84 percent of the global market. 
This is considered as 'chapter one', with only 20\% of data from enteprise moved to the cloud.


Chapter two requires a more complex technologies - in that “The other 80
[percent] now becomes not just more complex, it’s got a different complexion,”
she said, with customers wanting to keep their most important and sensitive data
on their own servers.

Companies established in the ‘pre-cloud’ era— such as IBM, Red Hat, Oracle and
Microsoft— stand to benefit from hybrid computing, as they can still sell their
software and hardware, as well as cloud services.




Enterprises are embracing Self-Service, and want to take advantage of multiple platforms and the choice and speed they provide. 
The main advantage of multiclouds
\begin{itemize}
  \item client can choose what service to run on the cloud that it fits (in terms of security-level, cost, easy-of-use)
\end{itemize}

several cloud management platform (CMP) 
\begin{enumerate}
  \item RightScale (started 2007, acquired by Flexa in 2018)
  
  \item ManageIQ (acquired by RedHat in 2012):
  
  provider of enterprise cloud management and automation solutions that enable
  organizations to deploy, manage and optimize private clouds, virtualized
  infrastructures and virtual desktops
  
  \item Cloudpia (acquired by Cisco in )
  
  \item Estratius (acquired by Dell in 2013)
  
  \item ServiceMesh (acquired by CSC in 2013)
  
  \item Nimbula (acquired by Oracle in 2013)
  
  \item ScaleXtreme (acquired by Citrix in 2014)
  
  \item Scalr:
  
  
Scalr is not an infrastructure provider or reseller. The infrastructure you
deploy on is yours: you give us the keys to your infrastructure cloud so we can
make the API calls to the provider on your behalf and so we can also rev up or
power down servers for you. When traffic piles up, Scalr detects the increased load, commissions new
  servers for you from the cloud, and then spreads the load.
  
  
  \item Terraform: Terraform is an open source too
  
  
Scalr and Terraform are primarily classified as "Cloud Management" and
"Infrastructure Build" tools respectively.
With Terraform, you describe your complete infrastructure as code, even as it
spans multiple service providers. Your servers may come from AWS, your DNS may
come from CloudFlare, and your database may come from Heroku.
Terraform will build all these resources across all these providers in parallel.
  
   \item RightScale:
   
   The multi-cloud integration enables you to choose your own clouds, providing freedom to work with any vendor in a rapidly changing market.
    
   \item Cloudify:
   
    an open source application management framework that allows users to manage even the most complex apps by automating their DevOps processes.
    
    
    \item Morpheus: 
    
    a cloud application management and orchestration platform that works on any cloud or infrastructure, from AWS to bare metal
    
  \item CopperEgg:
  
  Continuous visibility and cloud monitoring for all your servers – hosted or private, Linux or Windows. Works great with Amazon EC2, Rackspace, or any public or private cloud.
  
  \item Commando.io:
  
  A simpler way to manage servers online. Commando.io empowers users to be more
  efficient, improve their workflow, and eliminate anxiety over server
  provisioning, maintenance, and deployment.
  
  
  
\end{enumerate}

\section{RightScale}

RightScale, found in 2007, is one of the first multi-cloud management and cost
optimization companies.
It delivers vendor-agnostic cloud management capabilities, to bridge the gap between private and public clouds.
It was built as a platform to abstract provider-specific services, features, and
tools that could bring multiple IaaS environments into a common control plane.
Supported platforms:
\begin{itemize}
  \item initially: VMware, AWS and OpenStack
\end{itemize}


\chapter{Cloud-enabled applications}
\label{chap:cloud-native-apps}

In 2006, Amazon launched their Amazon Web Services (AWS) public cloud platform,
and a new era began. The era of cloud-native applications.

Application modernization is an area of great importance to your clients. Only
20 percent of applications and workloads that exist today are cloud-enabled


What does it mean to build and run an application on the cloud? There are 3
levels of cloud-service abstractions: Infrastructure as a Service (IaaS),
Platform as a Service (PaaS) and Software as a Service (SaaS).
 
\begin{enumerate}
  \item IaaS:
  giving users the basic infrastructure needed to build and deploy an application.
  
   Amazon Elastic Compute Cloud (EC2) falls into the IaaS category
   
   This typically requires the IaaS providers to provide PaaS.
  
  \item PaaS: 
  
  PaaS products offer a higher level of abstraction, so the user won’t be
  exposed to the O/S, middleware or runtime and needs only to concern him or
  herself with the application and data.
  
  This recently is being provided via containers.
  Within the PaaS market, two of the major players are Pivotal Cloud Foundry and
  Kubernetes. They are both open source cloud PaaS products for building,
  deploying and scaling applications.
  
  
  \item SaaS:
  SaaS products are applications built and hosted by a third-party platform and
  made available to users via the internet.
  
  The user can make query, e.g. REST APIs, and the cloud-native app runs,
  returns results via JSON file output.
  
  
  
\end{enumerate}
 
 