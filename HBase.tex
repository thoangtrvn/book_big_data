\chapter{HBase}
\label{chap:HBase}

HBase is a distributed NoSQL database that stores data in HDFS
(Sect.\ref{sec:HadoopFS}). HBase operations do not use MapReduce (Apache Hive is
the one that use MapReduce - Chap.\ref{chap:Hive_PrestoDB}) and thus provides a real-time
data access. Use it when you need  random, realtime read/write access to your
Big Data. This project's goal is the hosting of very large tables -- billions of
rows X millions of columns -- atop clusters of commodity hardware. HBase is
developed based on Google Bigtable (Chap.\ref{chap:google_bigtable}).

Based on Eric Brewer's CAP theorem, HBase is a CP type system (Consistency,
Partition Tolerance). HBase isn't fully ACID compliant, although it does support
certain properties. \url{https://www.xplenty.com/blog/2014/05/hive-vs-hbase/}

\section{Concepts}

HBase's data model is in the form of tables, consisting of rows and columns.
However, the concept of table in HBase is totally different from that in
relational database systems (MySQL, Oracle, \ldots). The concept of rows and
columns are also different. 

HBase is very much a distributed database; and it is more a ``Data Store'' than
a ``Data Base''  as it lacks many features found in a RDBMS (e.g. typed
columns, secondary indexes, triggers, advanced query languages\ldots). Languages
to query data stored in HBase is discussed in Sect.\ref{sec:query-HBase}.


{\bf WHEN TO USE HBase?}
\begin{enumerate}
  \item the data must be extremely large

Even HDFS doesn't do well with anything less than 5 DataNodes.
  
  \item your data can live outside RDBMS
  
  An application built against an RDBMS cannot be "ported" to HBase by simply
  changing a JDBC driver, for example. Consider moving from an RDBMS to HBase as
  a complete redesign as opposed to a port. 
  
\end{enumerate}


\url{http://hbase.apache.org/book/architecture.html}

\begin{enumerate}
  \item Table names are Strings and composed
of characters that are safe for use in a file system path.

  \item Row: Within a table, data is stored according to its row. Rows are identified
uniquely by their row key. Row keys do not have a data type and are always
treated as a byte[ ] (byte array).

Row keys are the equivalent of primary keys in relational
database tables. You cannot choose to change which column in an HBase table will
be the row key after the table has been set up

  \item Column Family: Data within a row is grouped by column family. Column
families also impact the physical arrangement of data stored in HBase. For this
reason, \textcolor{red}{they must be defined up front and are not easily
modified}.


Every row in a table has the same column families, although a row need not store
data in all its families. Column families are Strings and composed of characters that are safe
for use in a file system path.


  \item Column Qualifier: Data within a column family is addressed via its column
qualifier, or simply, column. Column qualifiers need not be specified in advance.


Column qualifiers need not be consistent between rows. Like row keys, column
qualifiers do not have a data type and are always treated as a byte[ ].

  \item Cell: A combination of row key, column family, and column qualifier uniquely
identifies a cell. The data stored in a cell is referred to as that cell's value. Values
also do not have a data type and are always treated as a byte[ ].

  \item Timestamp: Values within a cell are versioned. Versions are identified by their
version number, which by default is the timestamp of when the cell was written.
If a timestamp is not specified during a write, the current timestamp is used. If
the timestamp is not specified for a read, the latest one is returned. The number
of cell value versions retained by HBase is configured for each column family.
The default number of cell versions is three. 
\end{enumerate}

there are various ways of describing this data model. You can
view the same thing as if it's a key-value store, where the key
is the row key and the value is the rest of the data in a column.
You can also consider
HBase to be a key-value store where the key is defined as row key, column family,
column qualifier, timestamp, and the value is the actual data stored in the cell.

\textcolor{red}{In HBase, Key is
formed by [row key, column family, column qualifier, timestamp] and Value is the
contents of the cell.}

\section{Design a HBase table}

Designing HBase tables
can be defined as answering the following questions in the context of a use case:
\begin{verbatim}
1.	What should the row key structure be and what should it contain?
2.	How many column families should the table have?
3.	What data goes into what column family?
4.	How many columns are in each column family?
5.	What should the column names be? Although column names don't need to be
defined on table creation, you need to know them when you write or read data.
6.	What information should go into the cells?
7.	How many versions should be stored for each cell?
\end{verbatim}
In
order to define that effectively, it is important to define the access patterns (read
as well as write) up front



\section{Cluster structure for HBase}


Make sure that user name of all machines and the path where the hbase is
installed are same in all machines. There are 3 types of nodes
\begin{enumerate}
  \item HBase Master:  responsible for assigning regions to HbaseRegionserver,
  monitors the health of each HbaseRegionserver
  
  \item ZooKeeper: a centralized service for maintaining configuration
  information, naming, providing        distributed synchronization, and providing group services.
  
  \item HBase RegionServer:  responsible for handling client read and write
  requests. It communicates with the Hbasemaster to get a list of regions to serve and to tell the master that it is alive.
\end{enumerate}
We can configure one machine in the cluster is designated as Hbase master and
Zookeeper. These master processes may be collocated with the Hadoop NameNode and
ResourceManager (oldname: JobTracker) for small clusters.

\section{Install}  

You can either (1) downloaded the pre-compiled version, or (2) compile from
source. 

\subsection{Pre-built}

The pre-compiled version can be downloaded from
\url{http://www.apache.org/dyn/closer.cgi/hbase/}.
The latest version is 0.99.1 (Oct-2014) which is a "developer preview" release.

\subsection{Build from source}

It is more complicated when you build from source. I will walk through the
complexity behind building HBase. From HBase 0.96, the project is split into
multiple modules
\begin{verbatim}
hbase-annotations
hbase-assembly
hbase-checkstyle
hbase-client
hbase-common
hbase-examples
hbase-hadoop1-compat
hbase-hadoop2-compat
hbase-hadoop-compat
hbase-it
hbase-prefix-tree
hbase-protocol
hbase-rest
hbase-server
hbase-shell
hbase-testing-util
hbase-thrift
\end{verbatim}

With HBase 0.98.8, the default Hadoop 2 is 2.2.0. If you have Hadoop 2.5.2, you
need to modify pom.xml to use the new default Hadoop version.

First, you need to have Hadoop installed with HDFS (Chap.\ref{chap:Hadoop}) and
Java configured (Sect.\ref{sec:configure_Java}). Then, you can try with the
following command
\begin{verbatim}
mvn clean install
mvn clean install -DskipTests
\end{verbatim}
The problem is that \verb!Hbase-rest! requires root privilege to create some
folder (Sect.\ref{sec:Hbase-rest-test}). However, when we use \verb!sudo!, the
compiled modules are stored in the \verb!/root/.m2/repository!, instead of your current user home directory.

\begin{verbatim}
sudo mvn clean install
sudo mvn clean install -DskipTests
\end{verbatim}
If you're succeed, it will generate HBase artifacts and Maven POM files and
install them int your local Maven repository \verb!~/.m2/repository!
(Troubleshoot: Sect.\ref{sec:troubleshoot_build}). 

Now we generate binary tarball to install on the cluster. 
\begin{verbatim}
mvn package
mvn assembly:single
\end{verbatim}
which creates
\begin{verbatim}
./target/hbase-0.98.8-yourversion.tar.gz
\end{verbatim}
Finally, copy the tarball that was generated in the last step to your cluster
and unpack them in the desired location, e.g. /usr/local/hbase-0.98.8.
\url{http://praveen.kumar.in/2011/06/20/building-hadoop-and-hbase-for-hbase-maven-application-development/}

IMPORTANT: We can do all of them in a single command
\begin{verbatim}
mvn -DskipTests clean install package assembly:single

mvn -DskipTests clean install package && mvn -DskipTests assembly:single
\end{verbatim}

If there is any error, try to build modules one by one
(Sect.\ref{sec:hbase_build_one-by-one}).

\subsection{HBase build module-by-module}
\label{sec:hbase_build_one-by-one}

Based on the pom.xml, the build order is
\begin{verbatim}
[INFO] HBase - Annotations
[INFO] HBase - Common
[INFO] HBase - Protocol
[INFO] HBase - Client  (depends annotations,common,protocol)
[INFO] HBase - Hadoop Two Compatibility (i.e. hbase-hadoop-compat)
[INFO] HBase - Prefix Tree (depends annotations,commmon,hadoop-compat)
[INFO] HBase - Server (depends annotations,common,protocol,client,hadoop-compat) 
[INFO] HBase - Testing Util (depends annotations,common,protocol,
                                 client,server,hadoop-compat)
[INFO] HBase - Thrift (depends annotations,common,protocol,client,
                                 server,hadoop-compat,testing-util,
                                 ${compat.module} which can be hadoop1-compat or
                                                               hadoop2-compat
                      )
[INFO] HBase - Rest   (depends annotations,common,protocol,client,
                                 server,hadoop-compat,testing-util,
                                 ${compat.module} which can be hadoop1-compat or
                                                               hadoop2-compat
                      )
[INFO] HBase - Shell  (depends annotations,common,protocol,client,
                                 server,hadoop-compat,testing-util,
                                 ${compat.module} which can be hadoop1-compat or
                                                               hadoop2-compat
                      )
[INFO] HBase - Integration Tests (it)
[INFO] HBase - Examples (depends annotations,common,protocol,client,
                                 server,hadoop-compat,testing-util, thrift,
                         )
[INFO] HBase - Assembly  (depends annotations,common,protocol,client,
                                 server,hadoop-compat,testing-util,
                                 it,
                                 ${compat.module} which can be hadoop1-compat or
                                                               hadoop2-compat
                         )
\end{verbatim}
and the build cycle is
\begin{verbatim}
[validate, initialize, generate-sources, process-sources, generate-resources,
process-resources, compile, process-classes, generate-test-sources, 
process-test-sources, generate-test-resources, process-test-resources, 
test-compile, process-test-classes, test, prepare-package, package, 
pre-integration-test, integration-test, post-integration-test, 
verify, install, deploy]
\end{verbatim}
The important stages are \verb!package!, \verb!install! and \verb!deploy!. We
try the first module \verb!hbase-annotations! and \verb!package! command
\begin{verbatim}
mvn -e install -pl hbase-annotations -e -X 2>&1 | tee
       ../mhbase-annotations.txt
mvn -e install -pl hbase-common -e -X 2>&1 | tee
       ../mhbase-common.txt
mvn -e install -pl hbase-protocol -e -X 2>&1 | tee
       ../mhbase-protocol.txt
mvn -e install -pl hbase-client -e -X 2>&1 | tee
       ../mhbase-client.txt
mvn -e install -pl hbase-hadoop-compat -e -X 2>&1 | tee
       ../mhbase-hadoop-compat.txt
mvn -e install -pl hbase-hadoop2-compat -e -X 2>&1 | tee
       ../mhbase-hadoop2-compat.txt
mvn -e install -pl hbase-prefix-tree -e -X 2>&1 | tee
       ../mhbase-prefix-tree.txt       
mvn -e install -pl hbase-server -e -X 2>&1 | tee
       ../mhbase-server.txt
       
\end{verbatim}
NOTE: Do not use parallel threads (e.g. -T 8) during debug.
Check the output error in \verb!../mhbase-annotations.txt! file.

If you success, the result will be copied to 
\begin{verbatim}
~/.m2/repository/org/apache/hbase/hbase-annotations/0.98.8/
\end{verbatim}
Example:
\begin{verbatim}
[INFO] Installing /home/hadoop/hbase-0.98.8/hbase-annotations/target/hbase-annotations-0.98.8.jar to /home/hadoop/.m2/repository/org/apache/hbase/hbase-annotations/0.98.8/hbase-annotations-0.98.8.jar
[DEBUG] Writing resolution tracking file /home/hadoop/.m2/repository/org/apache/hbase/hbase-annotations/0.98.8/_maven.repositories
[INFO] Installing /home/hadoop/hbase-0.98.8/hbase-annotations/pom.xml to /home/hadoop/.m2/repository/org/apache/hbase/hbase-annotations/0.98.8/hbase-annotations-0.98.8.pom
[DEBUG] Writing resolution tracking file /home/hadoop/.m2/repository/org/apache/hbase/hbase-annotations/0.98.8/_maven.repositories
[DEBUG] Installing org.apache.hbase:hbase-annotations/maven-metadata.xml to /home/hadoop/.m2/repository/org/apache/hbase/hbase-annotations/maven-metadata-local.xml
[INFO] Installing /home/hadoop/hbase-0.98.8/hbase-annotations/target/hbase-annotations-0.98.8-tests.jar to /home/hadoop/.m2/repository/org/apache/hbase/hbase-annotations/0.98.8/hbase-annotations-0.98.8-tests.jar
[DEBUG] Writing resolution tracking file /home/hadoop/.m2/repository/org/apache/hbase/hbase-annotations/0.98.8/_maven.repositories
[DEBUG] Installing org.apache.hbase:hbase-annotations/maven-metadata.xml to /home/hadoop/.m2/repository/org/apache/hbase/hbase-annotations/maven-metadata-local.xml
\end{verbatim}


Using the \verb!-e -X! options, we will see that the following packages will be
created (for a full build)
\begin{verbatim}
[DEBUG] === REACTOR BUILD PLAN ================================================
[DEBUG] Project: org.apache.hbase:hbase-annotations:jar:0.98.8
[DEBUG] Tasks:   [package]
[DEBUG] Style:   Regular
[DEBUG] -----------------------------------------------------------------------
[DEBUG] Project: org.apache.hbase:hbase-common:jar:0.98.8
[DEBUG] Tasks:   [package]
[DEBUG] Style:   Regular
[DEBUG] -----------------------------------------------------------------------
[DEBUG] Project: org.apache.hbase:hbase-client:jar:0.98.8
[DEBUG] Tasks:   [package]
[DEBUG] Style:   Regular
[DEBUG] -----------------------------------------------------------------------
[DEBUG] Project: org.apache.hbase:hbase-hadoop2-compat:jar:0.98.8
[DEBUG] Tasks:   [package]
[DEBUG] Style:   Regular
[DEBUG] -----------------------------------------------------------------------
[DEBUG] Project: org.apache.hbase:hbase-prefix-tree:jar:0.98.8
[DEBUG] Tasks:   [package]
[DEBUG] Style:   Regular
[DEBUG] -----------------------------------------------------------------------
[DEBUG] Project: org.apache.hbase:hbase-server:jar:0.98.8
[DEBUG] Tasks:   [package]
[DEBUG] Style:   Regular
[DEBUG] -----------------------------------------------------------------------
[DEBUG] Project: org.apache.hbase:hbase-testing-util:jar:0.98.8
[DEBUG] Tasks:   [package]
[DEBUG] Style:   Regular
[DEBUG] -----------------------------------------------------------------------
[DEBUG] Project: org.apache.hbase:hbase-thrift:jar:0.98.8
[DEBUG] Tasks:   [package]
[DEBUG] Style:   Regular
[DEBUG] -----------------------------------------------------------------------
[DEBUG] Project: org.apache.hbase:hbase-rest:jar:0.98.8
[DEBUG] Tasks:   [package]
[DEBUG] Style:   Regular
[DEBUG] -----------------------------------------------------------------------
[DEBUG] Project: org.apache.hbase:hbase-shell:jar:0.98.8
[DEBUG] Tasks:   [package]
[DEBUG] Style:   Regular
[DEBUG] -----------------------------------------------------------------------
[DEBUG] Project: org.apache.hbase:hbase-it:jar:0.98.8
[DEBUG] Tasks:   [package]
[DEBUG] Style:   Regular
[DEBUG] -----------------------------------------------------------------------
[DEBUG] Project: org.apache.hbase:hbase-examples:jar:0.98.8
[DEBUG] Tasks:   [package]
[DEBUG] Style:   Regular
[DEBUG] -----------------------------------------------------------------------
[DEBUG] Project: org.apache.hbase:hbase-assembly:pom:0.98.8
[DEBUG] Tasks:   [package]
[DEBUG] Style:   Regular
[DEBUG] =======================================================================
[DEBUG] Thread pool size: 8
[INFO] Building with 8 threads
[DEBUG] Scheduling: MavenProject: org.apache.hbase:hbase-annotations:0.98.8 @ /home/hadoop/hbase-0.98.8/hbase-annotations/pom.xml
\end{verbatim}

Regardless of the module to be built, this is the process
\begin{enumerate}
  \item Use Maven enforcer plugin: to check Maven version, JDK version, O/S
  family, etc.
\begin{verbatim}
<configuration>
  <fail default-value="true">${enforcer.fail}</fail>
  <failFast default-value="false">${enforcer.failFast}</failFast>
  <ignoreCache default-value="false">${enforcer.ignoreCache}</ignoreCache>
  <project>${project}</project>
  <rules>
    <requireMavenVersion>
      <version>[3.0.3,)</version>
      <message> ... </message>
    </requireMavenVersion>
    <requireJavaVersion>
      <version>[1.6,)</version>
      <message> ... </message>
    </requireJavaVersion>
  </rules>
  <session>${session}</session>
  <skip default-value="false">${enforcer.skip}</skip>
</configuration>
\end{verbatim}

Output:
\begin{verbatim}
[DEBUG] Detected Maven Version: 3.0.5
[DEBUG] Detected Maven Version: 3.0.5 is allowed in the range [3.0.3,).
[DEBUG] Executing rule: org.apache.maven.plugins.enforcer.RequireJavaVersion
[DEBUG] Rule org.apache.maven.plugins.enforcer.RequireJavaVersion is cacheable.
[DEBUG] Detected Java String: 1.8.0_25
[DEBUG] Normalized Java String: 1.8.0-25
[DEBUG] Parsed Version: Major: 1 Minor: 8 Incremental: 0 Build: 25 Qualifier: null
[DEBUG] Detected JDK Version: 1.8.0-25 is allowed in the range [1.6,).

\end{verbatim}

  \item Use JaCoCo plugin for code coverage analysis at runtime. (see Maven
  chapter in Java manual)
  
  \item Retrieve JARs from remote repositories using Maven Remote Resources
  plugin
  
\end{enumerate}

\begin{Verbatim}
[DEBUG] Source directories: [/home/hadoop/hbase-0.98.8/hbase-annotations/src/main/java]
[DEBUG] Classpath: [/home/hadoop/hbase-0.98.8/hbase-annotations/target/classes
 /usr/lib/jvm/java-8-oracle/jre/../lib/tools.jar
 /home/hadoop/.m2/repository/com/github/stephenc/findbugs/findbugs-annotations/1.3.9-1/findbugs-annotations-1.3.9-1.jar
 /home/hadoop/.m2/repository/log4j/log4j/1.2.17/log4j-1.2.17.jar
 /home/hadoop/.m2/repository/junit/junit/4.11/junit-4.11.jar
 /home/hadoop/.m2/repository/org/hamcrest/hamcrest-core/1.3/hamcrest-core-1.3.jar]
\end{Verbatim}

\begin{verbatim}
Setting project property: settings.localRepository -> /home/hadoop/.m2/repository
Setting project property: org.apache.hbase:hbase-annotations:jar -> /home/hadoop/hbase-0.98.8/hbase-annotations/target/hbase-annotations-0.98.8.jar
Setting project property: org.apache.hbase:hbase-annotations:test-jar:tests -> /home/hadoop/hbase-0.98.8/hbase-annotations/target/hbase-annotations-0.98.8-tests.jar
Setting project property: maven.dependency.org.apache.hbase.hbase-annotations.jar.path -> /home/hadoop/hbase-0.98.8/hbase-annotations/target/hbase-annotations-0.98.8.jar
Setting project property: maven.dependency.jdk.tools.jdk.tools.jar.path -> /usr/lib/jvm/java-8-oracle/jre/../lib/tools.jar
Setting project property: maven.dependency.org.apache.hbase.hbase-annotations.tests.test-jar.path -> /home/hadoop/hbase-0.98.8/hbase-annotations/target/hbase-annotations-0.98.8-tests.jar

\end{verbatim}

\begin{verbatim}
Setting project property: localRepository ->        id: local
      url: file:///home/hadoop/.m2/repository/
   layout: none
\end{verbatim}

The version of Hadoop being used (which is important to modified the parameter
in pom.xml to get to the same version of Hadoop you want to deploy in the
cluster). With HBase 0.98.8, the default is Hadoop 2.2.0, so you may need to
change to a different value
\begin{verbatim}
Setting project property: maven.dependency.org.apache.hadoop.hadoop-common.jar.path -> /home/hadoop/.m2/repository/org/apache/hadoop/hadoop-common/2.5.2/hadoop-common-2.5.2.jar
Setting project property: maven.dependency.org.apache.hadoop.hadoop-annotations.jar.path -> /home/hadoop/.m2/repository/org/apache/hadoop/hadoop-annotations/2.5.2/hadoop-annotations-2.5.2.jar
Setting project property: maven.dependency.org.apache.hadoop.hadoop-auth.jar.path -> /home/hadoop/.m2/repository/org/apache/hadoop/hadoop-auth/2.5.2/hadoop-auth-2.5.2.jar
Setting project property: org.apache.hadoop:hadoop-common:jar -> /home/hadoop/.m2/repository/org/apache/hadoop/hadoop-common/2.5.2/hadoop-common-2.5.2.jar
Setting project property: org.apache.hadoop:hadoop-annotations:jar -> /home/hadoop/.m2/repository/org/apache/hadoop/hadoop-annotations/2.5.2/hadoop-annotations-2.5.2.jar
\end{verbatim}


And many other packages
\begin{verbatim}
Setting project property: jdk.tools:jdk.tools:jar -> /usr/lib/jvm/java-8-oracle/jre/../lib/tools.jar
Setting project property: com.google.guava:guava:jar -> /home/hadoop/.m2/repository/com/google/guava/guava/12.0.1/guava-12.0.1.jar
Setting project property: com.google.code.findbugs:jsr305:jar -> /home/hadoop/.m2/repository/com/google/code/findbugs/jsr305/1.3.9/jsr305-1.3.9.jar
Setting project property: commons-logging:commons-logging:jar -> /home/hadoop/.m2/repository/commons-logging/commons-logging/1.1.1/commons-logging-1.1.1.jar
Setting project property: commons-codec:commons-codec:jar -> /home/hadoop/.m2/repository/commons-codec/commons-codec/1.7/commons-codec-1.7.jar
Setting project property: commons-lang:commons-lang:jar -> /home/hadoop/.m2/repository/commons-lang/commons-lang/2.6/commons-lang-2.6.jar
Setting project property: commons-collections:commons-collections:jar -> /home/hadoop/.m2/repository/commons-collections/commons-collections/3.2.1/commons-collections-3.2.1.jar
Setting project property: commons-io:commons-io:jar -> /home/hadoop/.m2/repository/commons-io/commons-io/2.4/commons-io-2.4.jar
Setting project property: com.google.protobuf:protobuf-java:jar -> /home/hadoop/.m2/repository/com/google/protobuf/protobuf-java/2.5.0/protobuf-java-2.5.0.jar
Setting project property: commons-cli:commons-cli:jar -> /home/hadoop/.m2/repository/commons-cli/commons-cli/1.2/commons-cli-1.2.jar
Setting project property: org.apache.commons:commons-math3:jar -> /home/hadoop/.m2/repository/org/apache/commons/commons-math3/3.1.1/commons-math3-3.1.1.jar
Setting project property: xmlenc:xmlenc:jar -> /home/hadoop/.m2/repository/xmlenc/xmlenc/0.52/xmlenc-0.52.jar
Setting project property: commons-httpclient:commons-httpclient:jar -> /home/hadoop/.m2/repository/commons-httpclient/commons-httpclient/3.1/commons-httpclient-3.1.jar
Setting project property: commons-net:commons-net:jar -> /home/hadoop/.m2/repository/commons-net/commons-net/3.1/commons-net-3.1.jar
Setting project property: org.mortbay.jetty:jetty:jar -> /home/hadoop/.m2/repository/org/mortbay/jetty/jetty/6.1.26/jetty-6.1.26.jar
Setting project property: org.mortbay.jetty:jetty-util:jar -> /home/hadoop/.m2/repository/org/mortbay/jetty/jetty-util/6.1.26/jetty-util-6.1.26.jar
Setting project property: com.sun.jersey:jersey-core:jar -> /home/hadoop/.m2/repository/com/sun/jersey/jersey-core/1.8/jersey-core-1.8.jar
Setting project property: com.sun.jersey:jersey-json:jar -> /home/hadoop/.m2/repository/com/sun/jersey/jersey-json/1.8/jersey-json-1.8.jar
Setting project property: org.codehaus.jettison:jettison:jar -> /home/hadoop/.m2/repository/org/codehaus/jettison/jettison/1.3.1/jettison-1.3.1.jar
Setting project property: com.sun.xml.bind:jaxb-impl:jar -> /home/hadoop/.m2/repository/com/sun/xml/bind/jaxb-impl/2.2.3-1/jaxb-impl-2.2.3-1.jar
Setting project property: org.codehaus.jackson:jackson-jaxrs:jar -> /home/hadoop/.m2/repository/org/codehaus/jackson/jackson-jaxrs/1.8.8/jackson-jaxrs-1.8.8.jar
Setting project property: org.codehaus.jackson:jackson-xc:jar -> /home/hadoop/.m2/repository/org/codehaus/jackson/jackson-xc/1.8.8/jackson-xc-1.8.8.jar
Setting project property: com.sun.jersey:jersey-server:jar -> /home/hadoop/.m2/repository/com/sun/jersey/jersey-server/1.8/jersey-server-1.8.jar
Setting project property: asm:asm:jar -> /home/hadoop/.m2/repository/asm/asm/3.1/asm-3.1.jar
Setting project property: tomcat:jasper-compiler:jar -> /home/hadoop/.m2/repository/tomcat/jasper-compiler/5.5.23/jasper-compiler-5.5.23.jar
Setting project property: tomcat:jasper-runtime:jar -> /home/hadoop/.m2/repository/tomcat/jasper-runtime/5.5.23/jasper-runtime-5.5.23.jar
Setting project property: commons-el:commons-el:jar -> /home/hadoop/.m2/repository/commons-el/commons-el/1.0/commons-el-1.0.jar
Setting project property: net.java.dev.jets3t:jets3t:jar -> /home/hadoop/.m2/repository/net/java/dev/jets3t/jets3t/0.9.0/jets3t-0.9.0.jar
Setting project property: org.apache.httpcomponents:httpclient:jar -> /home/hadoop/.m2/repository/org/apache/httpcomponents/httpclient/4.1.2/httpclient-4.1.2.jar
Setting project property: org.apache.httpcomponents:httpcore:jar -> /home/hadoop/.m2/repository/org/apache/httpcomponents/httpcore/4.1.2/httpcore-4.1.2.jar
Setting project property: com.jamesmurty.utils:java-xmlbuilder:jar -> /home/hadoop/.m2/repository/com/jamesmurty/utils/java-xmlbuilder/0.4/java-xmlbuilder-0.4.jar
Setting project property: commons-configuration:commons-configuration:jar -> /home/hadoop/.m2/repository/commons-configuration/commons-configuration/1.6/commons-configuration-1.6.jar
Setting project property: commons-digester:commons-digester:jar -> /home/hadoop/.m2/repository/commons-digester/commons-digester/1.8/commons-digester-1.8.jar
Setting project property: commons-beanutils:commons-beanutils:jar -> /home/hadoop/.m2/repository/commons-beanutils/commons-beanutils/1.7.0/commons-beanutils-1.7.0.jar
Setting project property: commons-beanutils:commons-beanutils-core:jar -> /home/hadoop/.m2/repository/commons-beanutils/commons-beanutils-core/1.8.0/commons-beanutils-core-1.8.0.jar
Setting project property: org.slf4j:slf4j-api:jar -> /home/hadoop/.m2/repository/org/slf4j/slf4j-api/1.6.4/slf4j-api-1.6.4.jar
Setting project property: org.slf4j:slf4j-log4j12:jar -> /home/hadoop/.m2/repository/org/slf4j/slf4j-log4j12/1.7.5/slf4j-log4j12-1.7.5.jar
Setting project property: org.codehaus.jackson:jackson-core-asl:jar -> /home/hadoop/.m2/repository/org/codehaus/jackson/jackson-core-asl/1.8.8/jackson-core-asl-1.8.8.jar
Setting project property: org.codehaus.jackson:jackson-mapper-asl:jar -> /home/hadoop/.m2/repository/org/codehaus/jackson/jackson-mapper-asl/1.8.8/jackson-mapper-asl-1.8.8.jar
Setting project property: org.apache.avro:avro:jar -> /home/hadoop/.m2/repository/org/apache/avro/avro/1.7.4/avro-1.7.4.jar
Setting project property: com.thoughtworks.paranamer:paranamer:jar -> /home/hadoop/.m2/repository/com/thoughtworks/paranamer/paranamer/2.3/paranamer-2.3.jar
Setting project property: org.xerial.snappy:snappy-java:jar -> /home/hadoop/.m2/repository/org/xerial/snappy/snappy-java/1.0.4.1/snappy-java-1.0.4.1.jar
Setting project property: org.apache.hadoop:hadoop-auth:jar -> /home/hadoop/.m2/repository/org/apache/hadoop/hadoop-auth/2.5.2/hadoop-auth-2.5.2.jar
Setting project property: org.apache.directory.server:apacheds-kerberos-codec:jar -> /home/hadoop/.m2/repository/org/apache/directory/server/apacheds-kerberos-codec/2.0.0-M15/apacheds-kerberos-codec-2.0.0-M15.jar
Setting project property: org.apache.directory.server:apacheds-i18n:jar -> /home/hadoop/.m2/repository/org/apache/directory/server/apacheds-i18n/2.0.0-M15/apacheds-i18n-2.0.0-M15.jar
Setting project property: org.apache.directory.api:api-asn1-api:jar -> /home/hadoop/.m2/repository/org/apache/directory/api/api-asn1-api/1.0.0-M20/api-asn1-api-1.0.0-M20.jar
Setting project property: org.apache.directory.api:api-util:jar -> /home/hadoop/.m2/repository/org/apache/directory/api/api-util/1.0.0-M20/api-util-1.0.0-M20.jar
Setting project property: com.jcraft:jsch:jar -> /home/hadoop/.m2/repository/com/jcraft/jsch/0.1.42/jsch-0.1.42.jar
Setting project property: org.apache.zookeeper:zookeeper:jar -> /home/hadoop/.m2/repository/org/apache/zookeeper/zookeeper/3.4.6/zookeeper-3.4.6.jar
Setting project property: org.apache.commons:commons-compress:jar -> /home/hadoop/.m2/repository/org/apache/commons/commons-compress/1.4.1/commons-compress-1.4.1.jar

Setting project property: org.tukaani:xz:jar -> /home/hadoop/.m2/repository/org/tukaani/xz/1.0/xz-1.0.jar
Setting project property: org.apache.hadoop:hadoop-mapreduce-client-core:jar -> /home/hadoop/.m2/repository/org/apache/hadoop/hadoop-mapreduce-client-core/2.5.2/hadoop-mapreduce-client-core-2.5.2.jar
Setting project property: org.apache.hadoop:hadoop-yarn-common:jar -> /home/hadoop/.m2/repository/org/apache/hadoop/hadoop-yarn-common/2.5.2/hadoop-yarn-common-2.5.2.jar
Setting project property: org.apache.hadoop:hadoop-yarn-api:jar -> /home/hadoop/.m2/repository/org/apache/hadoop/hadoop-yarn-api/2.5.2/hadoop-yarn-api-2.5.2.jar
Setting project property: javax.xml.bind:jaxb-api:jar -> /home/hadoop/.m2/repository/javax/xml/bind/jaxb-api/2.2.2/jaxb-api-2.2.2.jar
Setting project property: javax.activation:activation:jar -> /home/hadoop/.m2/repository/javax/activation/activation/1.1/activation-1.1.jar
Setting project property: javax.servlet:servlet-api:jar -> /home/hadoop/.m2/repository/javax/servlet/servlet-api/2.5/servlet-api-2.5.jar
Setting project property: com.google.inject:guice:jar -> /home/hadoop/.m2/repository/com/google/inject/guice/3.0/guice-3.0.jar
Setting project property: javax.inject:javax.inject:jar -> /home/hadoop/.m2/repository/javax/inject/javax.inject/1/javax.inject-1.jar
Setting project property: aopalliance:aopalliance:jar -> /home/hadoop/.m2/repository/aopalliance/aopalliance/1.0/aopalliance-1.0.jar
Setting project property: com.sun.jersey.contribs:jersey-guice:jar -> /home/hadoop/.m2/repository/com/sun/jersey/contribs/jersey-guice/1.9/jersey-guice-1.9.jar
Setting project property: com.google.inject.extensions:guice-servlet:jar -> /home/hadoop/.m2/repository/com/google/inject/extensions/guice-servlet/3.0/guice-servlet-3.0.jar
Setting project property: io.netty:netty:jar -> /home/hadoop/.m2/repository/io/netty/netty/3.6.6.Final/netty-3.6.6.Final.jar
Setting project property: com.github.stephenc.findbugs:findbugs-annotations:jar -> /home/hadoop/.m2/repository/com/github/stephenc/findbugs/findbugs-annotations/1.3.9-1/findbugs-annotations-1.3.9-1.jar
Setting project property: log4j:log4j:jar -> /home/hadoop/.m2/repository/log4j/log4j/1.2.17/log4j-1.2.17.jar
Setting project property: junit:junit:jar -> /home/hadoop/.m2/repository/junit/junit/4.11/junit-4.11.jar
Setting project property: org.hamcrest:hamcrest-core:jar -> /home/hadoop/.m2/repository/org/hamcrest/hamcrest-core/1.3/hamcrest-core-1.3.jar
Setting project property: org.mockito:mockito-all:jar -> /home/hadoop/.m2/repository/org/mockito/mockito-all/1.9.0/mockito-all-1.9.0.jar

Setting project property: maven.dependency.org.apache.directory.server.apacheds-kerberos-codec.jar.path -> /home/hadoop/.m2/repository/org/apache/directory/server/apacheds-kerberos-codec/2.0.0-M15/apacheds-kerberos-codec-2.0.0-M15.jar
Setting project property: maven.dependency.org.apache.directory.server.apacheds-i18n.jar.path -> /home/hadoop/.m2/repository/org/apache/directory/server/apacheds-i18n/2.0.0-M15/apacheds-i18n-2.0.0-M15.jar
Setting project property: maven.dependency.org.apache.directory.api.api-asn1-api.jar.path -> /home/hadoop/.m2/repository/org/apache/directory/api/api-asn1-api/1.0.0-M20/api-asn1-api-1.0.0-M20.jar
Setting project property: maven.dependency.org.apache.directory.api.api-util.jar.path -> /home/hadoop/.m2/repository/org/apache/directory/api/api-util/1.0.0-M20/api-util-1.0.0-M20.jar
Setting project property: maven.dependency.com.jcraft.jsch.jar.path -> /home/hadoop/.m2/repository/com/jcraft/jsch/0.1.42/jsch-0.1.42.jar
Setting project property: maven.dependency.org.apache.zookeeper.zookeeper.jar.path -> /home/hadoop/.m2/repository/org/apache/zookeeper/zookeeper/3.4.6/zookeeper-3.4.6.jar
Setting project property: maven.dependency.org.apache.commons.commons-compress.jar.path -> /home/hadoop/.m2/repository/org/apache/commons/commons-compress/1.4.1/commons-compress-1.4.1.jar
Setting project property: maven.dependency.org.tukaani.xz.jar.path -> /home/hadoop/.m2/repository/org/tukaani/xz/1.0/xz-1.0.jar
Setting project property: maven.dependency.org.apache.hadoop.hadoop-mapreduce-client-core.jar.path -> /home/hadoop/.m2/repository/org/apache/hadoop/hadoop-mapreduce-client-core/2.5.2/hadoop-mapreduce-client-core-2.5.2.jar
Setting project property: maven.dependency.org.apache.hadoop.hadoop-yarn-common.jar.path -> /home/hadoop/.m2/repository/org/apache/hadoop/hadoop-yarn-common/2.5.2/hadoop-yarn-common-2.5.2.jar
Setting project property: maven.dependency.org.apache.hadoop.hadoop-yarn-api.jar.path -> /home/hadoop/.m2/repository/org/apache/hadoop/hadoop-yarn-api/2.5.2/hadoop-yarn-api-2.5.2.jar
Setting project property: maven.dependency.javax.xml.bind.jaxb-api.jar.path -> /home/hadoop/.m2/repository/javax/xml/bind/jaxb-api/2.2.2/jaxb-api-2.2.2.jar
Setting project property: maven.dependency.javax.activation.activation.jar.path -> /home/hadoop/.m2/repository/javax/activation/activation/1.1/activation-1.1.jar
Setting project property: maven.dependency.javax.servlet.servlet-api.jar.path -> /home/hadoop/.m2/repository/javax/servlet/servlet-api/2.5/servlet-api-2.5.jar
Setting project property: maven.dependency.com.google.inject.guice.jar.path -> /home/hadoop/.m2/repository/com/google/inject/guice/3.0/guice-3.0.jar
Setting project property: maven.dependency.javax.inject.javax.inject.jar.path -> /home/hadoop/.m2/repository/javax/inject/javax.inject/1/javax.inject-1.jar
Setting project property: maven.dependency.aopalliance.aopalliance.jar.path -> /home/hadoop/.m2/repository/aopalliance/aopalliance/1.0/aopalliance-1.0.jar
Setting project property: maven.dependency.com.sun.jersey.contribs.jersey-guice.jar.path -> /home/hadoop/.m2/repository/com/sun/jersey/contribs/jersey-guice/1.9/jersey-guice-1.9.jar
Setting project property: maven.dependency.com.google.inject.extensions.guice-servlet.jar.path -> /home/hadoop/.m2/repository/com/google/inject/extensions/guice-servlet/3.0/guice-servlet-3.0.jar
Setting project property: maven.dependency.io.netty.netty.jar.path -> /home/hadoop/.m2/repository/io/netty/netty/3.6.6.Final/netty-3.6.6.Final.jar
Setting project property: maven.dependency.com.github.stephenc.findbugs.findbugs-annotations.jar.path -> /home/hadoop/.m2/repository/com/github/stephenc/findbugs/findbugs-annotations/1.3.9-1/findbugs-annotations-1.3.9-1.jar
Setting project property: maven.dependency.log4j.log4j.jar.path -> /home/hadoop/.m2/repository/log4j/log4j/1.2.17/log4j-1.2.17.jar
Setting project property: maven.dependency.junit.junit.jar.path -> /home/hadoop/.m2/repository/junit/junit/4.11/junit-4.11.jar
Setting project property: maven.dependency.org.hamcrest.hamcrest-core.jar.path -> /home/hadoop/.m2/repository/org/hamcrest/hamcrest-core/1.3/hamcrest-core-1.3.jar
Setting project property: maven.dependency.org.mockito.mockito-all.jar.path -> /home/hadoop/.m2/repository/org/mockito/mockito-all/1.9.0/mockito-all-1.9.0.jar
\end{verbatim}



Other version settings that can be a good source for find incompatible
modules/packages
\begin{verbatim}
Setting project property: jamon-runtime.version -> 2.3.1
Setting project property: commons-cli.version -> 1.2
Setting project property: jetty.jspapi.version -> 6.1.14
Setting project property: commons-codec.version -> 1.7
Setting project property: it.test.jar -> hbase-it-0.98.8-hadoop2-tests.jar
Setting project property: surefire.skipFirstPart -> false
Setting project property: package.pid.dir -> /var/run/hbase
Setting project property: jacoco.version -> 0.6.2.201302030002
Setting project property: mockito-all.version -> 1.9.0
Setting project property: maven.build.timestamp.format -> yyyy-MM-dd'T'HH:mm
Setting project property: java.min.version -> 1.6
Setting project property: metrics-core.version -> 2.2.0
Setting project property: surefire.secondPartForkMode -> perThread
Setting project property: httpclient.version -> 3.1
Setting project property: project.build.sourceEncoding -> UTF-8
Setting project property: compat.module -> hbase-hadoop2-compat
Setting project property: test.output.tofile -> true
Setting project property: buildDate -> 2014-12-07T11:38
Setting project property: sourceReleaseAssemblyDescriptor -> source-release
Setting project property: surefire.firstPartParallel -> classes
Setting project property: package.log.dir -> /var/log/hbase
Setting project property: hbase-surefire.cygwin-argline -> -enableassertions -Xmx1900m -XX:MaxPermSize=256m
      -Djava.security.egd=file:/dev/./urandom -Djava.net.preferIPv4Stack=true
      "-Djava.library.path=${hadoop.library.path};/usr/java/packages/lib/amd64:/usr/lib64:/lib64:/lib:/usr/lib"
Setting project property: thrift.version -> 0.9.0
Setting project property: netty.version -> 3.6.6.Final
Setting project property: findbugs-maven-plugin.version -> 2.5.2
Setting project property: package.conf.dir -> /etc/hbase
Setting project property: commons-math.version -> 2.1
Setting project property: organization.logo -> http://www.apache.org/images/asf_logo_wide.gif
Setting project property: compileSource -> 1.6
Setting project property: commons-lang.version -> 2.6
Setting project property: joni.version -> 2.1.2
Setting project property: javadoc.version -> 2.9
Setting project property: argLine ->
Setting project property: common.test.jar -> hbase-common-0.98.8-tests.jar
Setting project property: surefire.firstPartForkMode -> once
Setting project property: hadoop-one.version -> 1.2.1
Setting project property: maven.site.version -> 3.3
Setting project property: collections.version -> 3.2.1
Setting project property: hadoop.version -> 2.5.2
Setting project property: gpg.useagent -> true
Setting project property: hbase-surefire.argLine -> -enableassertions -XX:MaxDirectMemorySize=1G -Xmx1900m
      -XX:MaxPermSize=256m -Djava.security.egd=file:/dev/./urandom -Djava.net.preferIPv4Stack=true
      -Djava.awt.headless=true
\end{verbatim}


\begin{verbatim}
Setting project property: assembly.file -> src/main/assembly/hadoop-two-compat.xml
Setting project property: package.prefix -> /usr
Setting project property: surefire.version -> 2.12-TRUNK-HBASE-2
Setting project property: jaxb-api.version -> 2.2.2
Setting project property: jersey.version -> 1.8
Setting project property: tar.name -> hbase-common-0.98.8.tar.gz
Setting project property: hbase.skip-jacoco -> true
Setting project property: guava.version -> 12.0.1
Setting project property: protobuf.version -> 2.5.0
Setting project property: jruby.version -> 1.6.8
Setting project property: maven.antrun.version -> 1.6
Setting project property: distMgmtSnapshotsName -> Apache Development Snapshot Repository
Setting project property: maven.min.version -> 3.0.3
Setting project property: server.test.jar -> hbase-server-0.98.8-tests.jar
Setting project property: package.release -> 1
Setting project property: htrace.version -> 2.04
Setting project property: build.platform -> Linux-amd64-64
Setting project property: jetty.version -> 6.1.26
Setting project property: arguments ->
Setting project property: jasper.version -> 5.5.23
Setting project property: slf4j.version -> 1.6.4
Setting project property: surefire.secondPartThreadCount -> 2
Setting project property: maven.assembly.version -> 2.4
Setting project property: surefire.firstPartGroups -> org.apache.hadoop.hbase.SmallTests
Setting project property: commons-io.version -> 2.4
Setting project property: distMgmtSnapshotsUrl -> https://repository.apache.org/content/repositories/snapshots
Setting project property: commons-logging.version -> 1.1.1
Setting project property: hadoop-two.version -> 2.5.2
Setting project property: jamon.plugin.version -> 2.3.4
Setting project property: surefire.provider -> surefire-junit47
Setting project property: surefire.skipSecondPart -> false
Setting project property: maven.resources.plugin.version -> 2.6
Setting project property: log4j.version -> 1.2.17
Setting project property: findbugs-annotations -> 1.3.9-1
Setting project property: surefire.secondPartGroups -> org.apache.hadoop.hbase.MediumTests
Setting project property: surefire.timeout -> 900
Setting project property: clover.version -> 2.6.3
Setting project property: hadoop-snappy.version -> 0.0.1-SNAPSHOT
Setting project property: jackson.version -> 1.8.8
Setting project property: final.name -> hbase-common-0.98.8
Setting project property: surefire.testFailureIgnore -> false
Setting project property: zookeeper.version -> 3.4.6
Setting project property: jettison.version -> 1.3.1
Setting project property: thrift.path -> thrift
Setting project property: surefire.firstPartThreadCount -> 1
Setting project property: junit.version -> 4.11
Setting project property: ant.file -> /home/hadoop/hbase-0.98.8/hbase-common/pom.xml

\end{verbatim}

\begin{verbatim}
Setting project property: project.groupId -> org.apache.hbase
Setting project property: project.artifactId -> hbase-common
Setting project property: project.name -> HBase - Common
Setting project property: project.description -> Common functionality for HBase
Setting project property: project.version -> 0.98.8
Setting project property: project.packaging -> jar
Setting project property: project.build.directory -> /home/hadoop/hbase-0.98.8/hbase-common/target
Setting project property: project.build.outputDirectory -> /home/hadoop/hbase-0.98.8/hbase-common/target/classes
Setting project property: project.build.testOutputDirectory -> /home/hadoop/hbase-0.98.8/hbase-common/target/test-classes
Setting project property: project.build.sourceDirectory -> /home/hadoop/hbase-0.98.8/hbase-common/src/main/java
Setting project property: project.build.testSourceDirectory -> /home/hadoop/hbase-0.98.8/hbase-common/src/test/java

\end{verbatim}

The following message is NOT an error, it just information that tells you to
ignore the quotes around the echoed command line and argument values
\begin{verbatim}
The ' characters around the executable and arguments are
not part of the command.
\end{verbatim}
\url{http://stackoverflow.com/questions/2338167/problem-with-ant-file}

The following message means the artifacts with incomplete POM metadata
\begin{verbatim}
[DEBUG] Supplemental data models won't be loaded.  No models specified.
[DEBUG] inceptionYear not specified, defaulting to 2014
\end{verbatim}
\url{http://maven.apache.org/plugins-archives/maven-remote-resources-plugin-1.3/xref/org/apache/maven/plugin/resources/remote/ProcessRemoteResourcesMojo.html}

\subsection{Build Hbase-rest}
\label{sec:Hbase-rest-test}

The HBase-rest has 2 failed tests
\begin{verbatim}
testGetClusterStatusXML()
testGetClusterStatusPB()
\end{verbatim}
with the error message
\begin{verbatim}
Running org.apache.hadoop.hbase.rest.TestStatusResource
Tests run: 2, Failures: 2, Errors: 0, Skipped: 0, Time elapsed: 106.332 sec <<< FAILURE!

Failed tests:   testGetClusterStatusXML(org.apache.hadoop.hbase.rest.TestStatusResource): 0>= 1
  testGetClusterStatusPB(org.apache.hadoop.hbase.rest.TestStatusResource): 0>= 1

\end{verbatim}

\begin{verbatim}
  @Test
  public void testGetClusterStatusXML() throws IOException, JAXBException {
    Response response = client.get("/status/cluster", Constants.MIMETYPE_XML);
    assertEquals(response.getCode(), 200);
    assertEquals(Constants.MIMETYPE_XML, response.getHeader("content-type"));
    StorageClusterStatusModel model = (StorageClusterStatusModel)
      context.createUnmarshaller().unmarshal(
        new ByteArrayInputStream(response.getBody()));
    validate(model);
  }

  @Test
  public void testGetClusterStatusPB() throws IOException {
    Response response = client.get("/status/cluster", Constants.MIMETYPE_PROTOBUF);
    assertEquals(response.getCode(), 200);
    assertEquals(Constants.MIMETYPE_PROTOBUF, response.getHeader("content-type"));
    StorageClusterStatusModel model = new StorageClusterStatusModel();
    model.getObjectFromMessage(response.getBody());
    validate(model);
    response = client.get("/status/cluster", Constants.MIMETYPE_PROTOBUF_IETF);
    assertEquals(response.getCode(), 200);
    assertEquals(Constants.MIMETYPE_PROTOBUF_IETF, response.getHeader("content-type"));
    model = new StorageClusterStatusModel();
    model.getObjectFromMessage(response.getBody());
    validate(model);

  }
\end{verbatim}

Strategy to debug (i.e. find out why HBase-rest's test case is not successful).
NOTE the function \verb!validate()! does the testing job. The function
\begin{verbatim}
assertTrue(java.lang.String message, boolean condition)
assertTrue(boolean condition)  
\end{verbatim}
test a condition, and print a message (for the first version) if the conditon is
not true.

\begin{enumerate}
  \item Comment out the \verb!validate()! function and return \verb!true!
\begin{verbatim}
private static void validate(StorageClusterStatusModel model) {
  assertTrue(true);
}  
\end{verbatim}

  \item Now, we look at the error message from above, it suggests this part of
  \verb!validate()! function fails
\begin{verbatim}
assertTrue(model.getRegions() + ">= 1", model.getRegions() >= 1);
\end{verbatim}
 
 Now, we go back, remove the \verb!assertTrue(true)! line, and comment out the
 line
\begin{verbatim}
//    assertTrue(model.getRegions() + ">= 1", model.getRegions() >= 1);
\end{verbatim}


\end{enumerate}

\subsection{Build Hbase-it}
\label{sec:Hbase-it-test}

The HBase-it 0.98.8 has 1 failed test and 1 error test
\begin{verbatim}
Running org.apache.hadoop.hbase.IntegrationTestIngestWithACL
Tests run: 1, Failures: 1, Errors: 0, Skipped: 0, Time elapsed: 710.409 sec <<< FAILURE!
Forking command line: /bin/sh -c cd /home/hadoop/hbase-0.98.8/hbase-it && /usr/lib/jvm/java-8-oracle/jre/bin/java -enableassertions -Xmx1900m -Djava.security.egd=file:/dev/./urandom -jar /home/hadoop/hbase-0.98.8/hbase-it/target/surefire/surefirebooter7679010813717069945.jar /home/hadoop/hbase-0.98.8/hbase-it/target/surefire/surefire4488376627296132998tmp /home/hadoop/hbase-0.98.8/hbase-it/target/surefire/surefire_136664039566412713544tmp

Results :

Failed tests:   testIngest(org.apache.hadoop.hbase.IntegrationTestIngestWithACL): Update failed with error code 1

Tests in error: 
  testBulkLoad(org.apache.hadoop.hbase.mapreduce.IntegrationTestBulkLoad): Unmatched family names found: unmatched family names in HFiles to be bulkloaded: [L, L, L, L, L, L, L, L, L, L, L, L, L, L, L, L]; valid family names of table IntegrationTestBulkLoad are: [D, S]

\end{verbatim}

Bug fix: \url{https://issues.apache.org/jira/browse/HBASE-10536}

Tests may not all pass so you may need to pass -DskipTests unless you are
inclined to fix the failing tests. A work-around solution is to skip the Test
\begin{verbatim}
$ mvn -DskipTests install  -pl hbase-it   -e -X 2>&1 | tee ../mhbase-it.txt
\end{verbatim}

\subsection{Resume the build}


Suppose the build stop at \verb!hbase-it! due to some errors, then after fixing
it, we can continue with \verb!<goals>! is either \verb!compile!, or
\verb!install! or \ldots
\begin{verbatim}
mvn <goals> -rf :hbase-it
\end{verbatim}

\section{Unit Testing in HBase}

\url{http://blog.cloudera.com/blog/2013/09/how-to-test-hbase-applications-using-popular-tools/}

Unit testing in  HBase application uses JUnit, Mockito, MRUnit, and
HBaseTestingUtility. HBaseTestingUtility is a mini-cluster
(Sect.\ref{sec:HBase_mini-cluster}) for integration
testing (Sect.\ref{sec:HBase_integration_test}).

\subsection{HBase mini-cluster}
\label{sec:HBase_mini-cluster}

\verb!HBaseTestingUtility! class is used to create an
instance which is used to keep it around doing HBase testing (see HBase-rest test cases).
\begin{verbatim}
public class TestStatusResource {
  private static final HBaseTestingUtility TEST_UTIL = new HBaseTestingUtility();
  private static final HBaseRESTTestingUtility REST_TEST_UTIL = 
    new HBaseRESTTestingUtility();
  ...
}
\end{verbatim}
\url{http://hbase.apache.org/book/ch19s04.html}

The class definition is
\begin{verbatim}
./hbase-server/src/test/java/org/apache/hadoop/hbase/HBaseTestingUtility.java
\end{verbatim}

\begin{verbatim}
/* mini-cluster of HBase, no DFS, Zookeeper
 public MiniHBaseCluster startMiniCluster() throws Exception {
                         // (#master, #slaves)
    return startMiniCluster(1, 1);
  }

  public MiniHBaseCluster startMiniCluster(final int numMasters,
    final int numSlaves)
  throws Exception { 
             // third argument = information of the DataNode-Hosts
    return startMiniCluster(numMasters, numSlaves, null);
  }                              
  
  // dataNodeHosts[] = the list of string-(hostname/IPs) of the dataNodes
  // if dataNodeHosts == null, the # of datanodes to start is null as well
  //                    else   the  # of datanodes to start is given in
  //                             array length dataNodeHost.length
 public MiniHBaseCluster startMiniCluster(final int
  numMasters, final int numSlaves, final String[] dataNodeHosts) throws Exception {
    return startMiniCluster(numMasters, numSlaves, numSlaves, dataNodeHosts, null, null);
  }                                                                                        
  
  /*  masterClass = the class to use as HMaster (default:null)
      regionserverclass = the class to use as HRegionServer (default:null)
  */
  public MiniHBaseCluster startMiniCluster(final int numMasters,
      final int numSlaves, final String[] dataNodeHosts, Class<? extends HMaster> masterClass,
      Class<? extends MiniHBaseCluster.MiniHBaseClusterRegionServer> regionserverClass)
          throws Exception {
    return startMiniCluster(
        numMasters, numSlaves, numSlaves, dataNodeHosts, masterClass, regionserverClass);
  }         
\end{verbatim}
starts up  a minicluster of hbase, optionally dfs, and zookeeper. A cluster data
directory is named randomly under \verb!System! property \verb!test.build.data!.
This directory is cleaned up on exit. 

The last class does
\begin{verbatim}
// (1)
// print out
    LOG.info("Starting up minicluster with " + numMasters + " master(s) and " +                                                                                            
        numSlaves + " regionserver(s) and " + numDataNodes + " datanode(s)");                                                                                              
                                                                                
// (2)
//  configure cluster TestDir
   setupClusterTestDir();
   System.setProperty(TEST_DIRECTORY_KEY, this.clusterTestDir.getPath());
                                                                                
// (3)
// run DFS
    startMiniDFSCluster(numDataNodes, dataNodeHosts);

// (4)
// zookeeper
    if (this.zkCluster == null) {
      startMiniZKCluster(clusterTestDir);
    }

// (5) final
// Start the MiniHBaseCluster
    return startMiniHBaseCluster(numMasters, numSlaves, masterClass, regionserverClass);

\end{verbatim}


\begin{mdframed}
To interact with the distributed or mini cluster
uniformly, \verb!IntegrationTestingUtility!, and \verb!HBaseCluster! classes,
and public client API's can be used. 

IMPORTANT: The use of 
\begin{verbatim}
HBaseClusterTestCase
HBaseTestCase
\end{verbatim}
are deprecated. We should use \verb!junit4! and \verb!HBaseTestingUtility!
(provided by HBase-testing-util module).

\end{mdframed}


\subsection{Introduce log in the Unit Tests}

\begin{verbatim}
import org.apache.commons.logging.Log; 
import org.apache.commons.logging.LogFactory; 

@Category(MediumTests.class)                                                                                                                                              
public class TestStatusResource {                                                                                                                                         
     static final Log LOG = LogFactory.getLog(TestStatusResource.class);
     
     ...
}
\end{verbatim}
then in the Unit Test code inside TestStatusResource class, we just add
\begin{verbatim}
    LOG.info(" string" + anything here to be printed out); 
\end{verbatim}


\subsection{HBase integration tests}
\label{sec:HBase_integration_test}

HBase integration/system tests are tests that are beyond HBase unit tests. This
takes the most of time.
Integration tests can be run in two modes: using a mini cluster, or against an
actual distributed cluster. Here, HBase use mini cluster
(Sect.\ref{sec:HBase_mini-cluster}).
%  and requires a mini-cluster or a distributed
% cluster.


Integration tests currently live under the src/test directory in the \verb!hbase-it! submodule.
All integration tests are also annotated with
\verb!@Category(IntegrationTests.class)!.


\begin{verbatim}
./hbase-rest/src/test/java/org/apache/hadoop/hbase/rest/HBaseRESTTestingUtility.java
\end{verbatim}

Another important class
\begin{verbatim}
./hbase-rest/src/main/java/org/apache/hadoop/hbase/rest/model/StorageClusterStatusModel.java
\end{verbatim}
which is used to unmarshal an XML representation of a class
\begin{verbatim}
@XmlRootElement(name="ClusterStatus")
@InterfaceAudience.Private
public class StorageClusterStatusModel 
    implements Serializable, ProtobufMessageHandler {
  private static final long serialVersionUID = 1L;

}
\end{verbatim}

\section{Troubleshoot}
\label{sec:troubleshoot_build}

From HBase 0.98, Hadoop 2.x is the default. To build HBase again a particular
version of Hadoop, we use \verb!hadoop.profile! property
\begin{verbatim} 
sudo mvn -Dhadoop.profile=1.0 ....
\end{verbatim}
However, the default version in 2.x profile is 2.2.0 (pom.xml)
\begin{verbatim}
<hadoop-two.version>2.2.0</hadoop-two.version>
\end{verbatim}
You may get Hbase-rest error when you try to compile the HBase code with HDFS
2.5.2. To resolve the problem, modify it to 
\begin{verbatim}
<hadoop-two.version>2.5.2</hadoop-two.version>
\end{verbatim}

If you get the error
\begin{verbatim}
was cached in the local repository, resolution will not be reattempted until
 the update interval of apache release has elapsed or updates are forced 
\end{verbatim}
you can run with \verb!-U! option 
\begin{verbatim}
sudo mvn clean compile -U
\end{verbatim}
Another solutions
\begin{itemize}
  \item  deleting the corresponding failed to download artifact directory in my
  local repo, and rerun the Maven
\end{itemize}
\url{http://stackoverflow.com/questions/4856307/when-maven-says-resolution-will-not-be-reattempted-until-the-update-interval-of}


You may get the error
\begin{verbatim}
Failed to execute goal on project hbase-assembly: Could not resolve dependencies
for project org.apache.hbase:hbase-assembly:pom:0.98.8: Could not find artifact
org.apache.hbase:hbase-it:jar:tests:0.98.8 in maven release (https://repo1.maven.org/maven2/) -> [Help 1]  

org.apache.maven.lifecycle.LifecycleExecutionException: Failed to execute goal
on project hbase-assembly: Could not resolve dependencies for project 
org.apache.hbase:hbase-assembly:pom:0.98.8: Could not find artifact
org.apache.hbase:hbase-it:jar:tests:0.98.8 in maven release
(https://repo1.maven.org/maven2/)   
\end{verbatim}
The reason is that the file
\begin{verbatim}
     <dependency>
        <artifactId>hbase-it</artifactId>
        <groupId>org.apache.hbase</groupId>
        <version>${project.version}</version>
        <type>test-jar</type>
        <scope>test</scope>
      </dependency>

 <it.test.jar>hbase-it-${project.version}-tests.jar</it.test.jar> 
\end{verbatim}
cannot be found on the above repositories. The solution is (1) download it to
local (cached) repository (see the instruction on Maven Chapter) by putting the file
\begin{verbatim}
 ~/.m2/repository/org/apache/hbase/hbase-it/0.98.8/hbase-it-0.98.8-tests.jar
\end{verbatim}
or, (2) add a new repository that has the file to pom.xml. 
\begin{verbatim}
    <repository>
      <id>apache release</id>
      <url>https://repo1.maven.org/maven2/</url>
    </repository>
\end{verbatim}
which allow Maven to try downloading the file as
\begin{verbatim}
https://repo1.maven.org/maven2/org/apache/hbase/hbase-it/0.98.8/hbase-it-0.98.8.pom
https://repo1.maven.org/maven2/org/apache/hbase/hbase-it/0.98.8/hbase-it-0.98.8-test.jar
\end{verbatim}

Unfortunately, using that repository, the path to the file is
\begin{verbatim}
https://repo1.maven.org/maven2/org/apache/hbase/hbase-it/0.98.8-hadoop2/hbase-it-0.98.8-hadoop2.pom
\end{verbatim}
which means we need to update the version for the package hbase-it-test from
\begin{verbatim}
<it.test.jar>hbase-it-${project.version}-tests.jar</it.test.jar> 
\end{verbatim}
to 
\begin{verbatim}
<it.test.jar>hbase-it-${project.version}-hadoop2-tests.jar</it.test.jar> 
\end{verbatim}


\begin{mdframed}
To understand more about Maven (\verb!mvn! command), we can read Java manual
book (Chapter on Maven). After \verb!mvn! is the target name (e.g. package, or
compile). We can define as many targets as we want. The \verb!compile! target
does not create .JAR files.
Potential errors is discussed in Sect.\ref{sec:hbase0.98_java8}.

\end{mdframed}

If you get the error
\begin{verbatim}
ERROR Failed to execute goal
org.apache.maven.plugins:maven-assembly-plugin:2.4:single (default-cli) on
project hbase-assembly: Failed to create assembly: Artifact:
org.apache.hbase:hbase-checkstyle:jar:0.98.7 (included by module) does not have
an artifact with a file.    
\end{verbatim}
NOTE: HBase does not use \verb!maven-assembly-plugin! to build in the main
project, instead it is used in the \verb!hbase-assembly! module.
\begin{verbatim}
<!-- in the main projec pom.xml -->
 <build>
    <plugins>
       <plugin>                                                                                                                                                           
          <artifactId>maven-assembly-plugin</artifactId>                                                                                                                   
          <version>${maven.assembly.version}</version>                                                                                                                     
          <configuration>                                                                                                                                                  
            <!--Defer to the hbase-assembly sub-module.  It                                                                                                                
             does all assembly-->                                                                                                                                          
            <skipAssembly>true</skipAssembly>                                                                                                                              
            <!--Do not attach assembly to project.-->                                                                                                                      
            <attach>false</attach>                                                                                                                                         
          </configuration>                                                                                                                                                 
        </plugin>                             
    </plugins>
</build>    
\end{verbatim}

\subsection{HBase 0.98 and Java 8}
\label{sec:hbase0.98_java8}

There is a deprecated method that gives the error
\begin{verbatim}
PoolMap.java:[100,17] error: name clash: remove(K,V) in PoolMap and
       remove(Object,Object) in Map 
\end{verbatim}
use patch:
\url{https://issues.apache.org/jira/secure/attachment/12622579/HBASE-10327.patch}


\section{Modules in Hbase}

\subsection{hbase-checkstyle}

Module to hold Checkstyle properties in HBase.
\url{http://maven-repository.com/artifact/org.apache.hbase/hbase-checkstyle}


\subsection{HBase-thrift}

HBase-thrift use \verb!org.cloudera.htrace! (check pom.xml of hbase-thrift) 
\begin{verbatim}
 <dependency>
      <groupId>org.apache.hbase</groupId>
      <groupId>org.cloudera.htrace</groupId>       
      <artifactId>htrace-core</artifactId>
 </dependency>         
\end{verbatim}
HTrace is a tracing framework intended for use with distributed systems written
in java.

{\bf IMPORTANT}: From December, 2014, HTrace has moved to Apache Incubator
\begin{verbatim}
The project WAS hosted here at http://github.com/cloudera/htrace but as of
December 2014, it has moved to Apache Incubator. Go to
http://incubator.apache.org/projects/htrace and
http://htrace.incubator.apache.org from here on out.  

The old state of the project is available in Maven Central with groupId:
org.htrace, and name: htrace. The Apache-hosted htrace will have a groupId of
org.apache.htrace and name htrace. (It was formally at groupId:
org.cloudera.htrace, and name: htrace).  
\end{verbatim}


\subsection{hbase-rest}
 
HBase REST exposes HBase tables, rows, cells, and metadata as URL specified
resources via the REST APIs. For any problem when compiling with testing not
successful, read the Chapter Web Services on RESTful web service on Java manual book.

HBase REST provides a RESTful Web service front end for HBase (Check Java
manual book on RESTful web service) (Chap.\ref{chap:REST}).
\url{https://hbase.apache.org/apidocs/org/apache/hadoop/hbase/rest/package-summary.html}

% 
% In architectural design, there are 2 approaches: (1) start from nothing and
% build it, (2) start with a system as a whole without constraints, and try to
% build the relations between the components. Where the first emphasizes
% creativity and unbounded vision, the second emphasizes restraint and
% understanding of the system context. {\bf REST} has been developed using the
% second process.
% \url{http://www.ics.uci.edu/~fielding/pubs/dissertation/rest_arch_style.htm}
% 
% RESTful web service 



\subsection{hbase-annotations}

Define the annotations for test categories (lightweight, medium, heavy tests)
\begin{verbatim}
 hbase-annotations/src/test/java/org/apache/hadoop/hbase/
   IntegrationTests.java  
   LargeTests.java        
   MediumTests.java      
   SmallTests.java   
\end{verbatim}

Since HBase 0.99.0+, they are moved to
\begin{verbatim}
 hbase-annotations/src/test/java/org/apache/hadoop/hbase/testclassification
 
\end{verbatim}

\section{Large-scale deployment}

Facebook: \url{https://www.facebook.com/note.php?note_id=454991608919}
 
\section{Operations in HBase}


HBase works by storing data as key/value. It supports four primary operations:
put to add or update rows, scan to retrieve a range of cells, get to return
cells for a specified row, and delete to remove rows, columns or column versions
from the table. Versioning is available so that previous values of the data can
be fetched (the history can be deleted every now and then to clear space via
HBase compactions).     

Although HBase includes tables, a schema is only required for tables and column
families, but not for columns, and it includes increment/counter functionality. 
 
\section{Query language: HBase API}
\label{sec:query-HBase}

Data that can be accessed via HBase is stored in HDFS file system.
Thus, this data stored can be accessed using MapReduce jobs via Hadoop.

Other than that, HBase provides Java client APIs (to access HBase data from your
Java program); and other non-Jave front-ends via gateway APIs: such as REST,
Avro or Thrift. 

For high volume query optimization, HBase also support Block Cache and Bloom
Filters. 

However, an SQL-like query language can be used if Apache Phoenix is installed (Sect.\ref{sec:Phoenix}).

HBase provides build-in web-pages for operational insight as well as JMX
metrics.

HBase's API for data manipulation consists of three primary methods: Get, Put,
and Scan.
\begin{itemize}
  \item Gets and
Puts are specific to particular rows and need the row key to be provided
  \item Scans are
done over a range of rows.

The range could be defined by a start and stop row key or
could be the entire table if no start and stop row keys are defined.
\end{itemize}


Now
\begin{enumerate}
  \item If you were to retrieve the item that the row key maps to, you'd get data from
all the columns back
 
  \item If you were to retrieve the item that a particular column
family maps to, you'd get back all the column qualifiers and the associated maps.

  \item If you were to retrieve the item that a particular column qualifier maps to, you'd get
all the timestamps and the associated values.

HBase optimizes for typical patterns and returns only the latest version by default.
You can request multiple versions as a part of your query.

  \item  
\end{enumerate}

There are various ways to access and interact with Apache HBase. 

\subsection{REST interface}

\subsection{Thrift interface}


Thrift is both cross-platform and more lightweight than REST for many
operations. \url{http://wiki.apache.org/hadoop/Hbase/ThriftApi}


\section{Apache Phoenix}
\label{sec:Phoenix}

Apache Phoenix is a relational database layer over HBase delivered as a
client-embedded JDBC driver targeting low latency queries over HBase data. 

Apache Phoenix takes your SQL query, compiles it into a series of HBase scans,
and orchestrates the running of those scans to produce regular JDBC result sets.

\url{http://phoenix.apache.org/}
