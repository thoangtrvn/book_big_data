\chapter{Solr}
\label{chap:Solr}

Solr is the descendant of {\bf Lucene}, a search indexer for information
retrieval (Sect.\ref{sec:Lucene}). Perhaps the most significant deployment of
Lucene is Wikipedia, where it powers search for the entire site.

\section{Lucene (information retrieval)}
\label{sec:Lucene}


Lucene is an open-source {\it information retrieval} software library
(Chap.\ref{chap:Information_Retrieval}), initially written in Java. It has been
ported to  Delphi, Perl, C\#, C++, Python, Ruby, and PHP.

While suitable for any applications that require {\it full-text indexing} and
searching capability, Lucene has been widely used mainly in implementation of
{\bf Internet search engines} and local, single-site searching. 

A single document is represented as {\bf fields} of text. So, the APIs of Lucene
is independent of the file format. It can accepts text from PDF, HTML, MS Word,
OpenDocuments and many others (as long as the textual information can be
extracted); but \textcolor{red}{not images}.

Lucene needs user to feed the data into the system. So, it doesn't do crawling
and HTML parsing. There are other projects that use Lucene and add these
features.
\begin{enumerate}
  \item Apache Nutch: web crawling and HTML parsing
  \item Apache Solr: enterprise search server (Sect.\ref{sec:Solr})
  \item Elasticsearch: enterprise search server (Chap.\ref{chap:elastic_search})
  \item DocFetcher: multi-platform desktop search application
  (Sect.\ref{sec:DocFetcher})
  \item 
\end{enumerate}

Similar tools (NOTE: Lucene is written in Java)
\begin{enumerate}
  \item Lucene.NET: .NET framework
  \item Ferret: Ruby language, and use Poshlib
  \item Kinosearch: Perl and C
  \item Apache Lucy: the successor of Ferret and Kinosearch, with bindings to
  both Perl and Ruby.
\end{enumerate}

A GUI to Lucene is 
\begin{enumerate}
  \item {\bf Luke}: Java-based GUI (display and modify indexes)
\end{enumerate}

Companies that use Lucene
\begin{enumerate}
  \item Twitter: real-time search
\end{enumerate}

\section{Solr}
\label{sec:Solr}

Solr was created in 2004 at CNET Networks as an in-house project to add search
capability for the company website. Every new open-source project need to go
through an incubation period to ensure the open-source is legal; when it
passes, it is graduated. Solr started the incubation period in Jan, 2006 and
graduated in Jan, 2007.

Derived from Lucene (Sect.\ref{sec:Lucene}), Solr enables the search and
indexing as  a {\it search platform} for being used in an enterprise.
Solr is highly scalable, with support for NoSQL features (Sect.\ref{sec:NoSQL}).
It also includes full-text search, hit highlighting, faceted search, dynamic
clustering, database integration, and rich document (e.g., Word, PDF) handling.
Solr's powerful external configuration allows it to be tailored to many types of
application without Java coding, and it has a plugin architecture to support
more advanced customization.

Solr uses the Lucene Java search library at its core for full-text indexing and
search, and has REST-like HTTP/XML and JSON APIs that make it usable from most
popular programming languages.

\begin{enumerate}
  \item Solr 1.3 (2008, Sept): distributed search capabilities and performance
  enhancements 
  \item Solr 1.4 (2008, Nov): nhancements in indexing, searching and faceting
  along with many other improvements such as Rich Document processing (PDF,
  Word, HTML), Search Results clustering based on Carrot2 (Sect.\ref{sec:Carrot})
  and also improved database integration. The release also features many additional plug-ins   
  
  \item 2010, March: Solr and Lucene merged
  \item Solr 3.1 (2011): version scheme changed to match with Lucene
  \item Solr 4.0 (2012, Oct): new SolrCloud feature
  \item Solr 4.1, 4.2, 4.2.1
\end{enumerate}


\section{Carrot2}
\label{sec:Carrot}

Carrot is an open-source search result clustering engine.
Carrot$^2$ offers ready-to-use components for fetching search results from
various sources.

Typically, it fetch search results from an external search engine
\begin{verbatim}
Bing search API
PubMed
Lucene index
OpenSearch
Solr server
Generic XML files
eTools metasearch
\end{verbatim}
Lucene / Solr index or load text files from a local disk. We can add support for
others using code examples provided.
