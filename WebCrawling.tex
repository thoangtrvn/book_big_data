\chapter{Web Crawling}
\label{chap:WebCrawling}

\section{Introduction}


\subsection{screen scraping}
\label{sec:screen_scrapping}

Originally, screen scrapping refers to the practice of reading text data from
the terminal's screen, by reading the terminal's memory through its auxiliary
port. UI from that area were often simply text-based dumped terminals.
The screen scraper might connect to the legacy system via Telnet, emulate the
keystrokes needed to navigate the old user interface, process the resulting
display output, extract the desired data, and pass it on to the modern system.




\subsection{web scraping}
\label{sec:web_scrapping}

Even though web pages (built from HTML, XHTML) requently contain a wealth of useful data in text form, most have been designed for human end-users, and not for ease of automated use.
Several tools to scrape web content (known as {\bf web scraper}) have been developed.
A web scraper provides APIs to extract data from a web site.

Newer forms of web scraping involve listening to data feeds from web servers, e.g. feeds in JSON format.
Recently, these scrapers rely on using techniques in DOM parsing, computer vision and natural language processing
that can simulate human processing that occurs when reading a webpage, to automatically extract useful information.

\subsection{Tools}

Scrapy (Python-based tool - Sect.\ref{sec:Scrapy}) which is faster than
Mechanize but not as scalable as Nutch (Chap.\ref{chap:Nutch}) or Heritrix,
which means that it's not meant to be used for crawling the entire web, but it's
OK for crawling a lot (5000+) of sites, even huge ones like Amazon.



\section{Scrapy (Python-based)}
\label{sec:Scrapy}

So you need to extract some information from a website, but the website doesn't
provide any API or mechanism to access that info programmatically. Scrapy can
help you extract that information.

Scrapy is an application framework for crawling web sites and extracting
structured data which can be used for a wide range of useful applications, like
data mining, information processing or historical archival.


We should read this in Python manual book.


\section{Crawler4j (Java-based)}

You can setup a multi-threaded web crawler in few minutes.

A crawler is a class extend from \verb!WebCrawler! class, with 3 informations need to be provided
\begin{enumerate}
  \item the pattern
  \item override \verb!shoudVisit()! method: decides whether the given URL should be crawled or not.
  \item override \verb!visit()! method: is called after the content of a URL is downloaded successfull. What it does is
  to extract the right information you want ( url, text, links, html, and unique id of the downloaded page)
\end{enumerate}

A controller class is also required, from which you provide the seed URLs.

Other settings:
\begin{enumerate}
  \item depth of search: \verb!crawlConfig.setMaxDepthOfCrawling(maxDepthOfCrawling);!
  
  \item max number of pages to crawl: \verb!crawlConfig.setMaxPagesToFetch(maxPagesToFetch);!
  
  \item politeness, i.e. how often to request a page to a website (default: 200 ms): \verb!crawlConfig.setPolitenessDelay(politenessDelay);!
  
  \item run behind proxy or not?
  
  \item resume crawling or not?
  
  \item agent string: a user agent strings identify what a client (e.g. a web browser, a crawler) is using to access a web-page.
  
  \url{http://whatsmyuseragent.com/WhatsAUserAgent}
\end{enumerate}

\begin{verbatim}
public class MyCrawler extends WebCrawler {

    private final static Pattern FILTERS = Pattern.compile(".*(\\.(css|js|gif|jpg"
                                                           + "|png|mp3|mp3|zip|gz))$");
                                                           
                                                           
\end{verbatim}
\url{https://github.com/yasserg/crawler4j}