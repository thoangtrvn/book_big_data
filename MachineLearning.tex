
 

\chapter{Machine learning}
\label{sec:machine-learning}

Machine learning allows {\it a machine to find hidden insights without being
explicitly programmed where to look for}. See applications
(Sect.\ref{sec:application-machine-learning}).

In a traditional way, there are several applications for Machine Learning
(ML), 
\begin{enumerate}
  \item  data mining
   
  \item classification
\end{enumerate}

The classification is a fundamental problem in that machine learning techniques
have been widely used to solve;  e.g. neural networks. 


\section{Abstract a problem in mathematical language}

The most common mathematical framework for learning.
\begin{enumerate}
  \item training samples

$D$ contains a set of data item
\begin{equation}
D = {z_1, z_2, \ldots, z_n}
\end{equation}
with $z_i$ represent a completed data item and is sampled from an {\bf
unknown} process $P(Z)$. \textcolor{red}{REMEMBER that the goal is to build a
system $f$ that is (trained to behave/producing output/making decision) as
closed to (as good as) this unknown process as possible}.
  
  \item A loss functional $L$: 

\begin{equation}
L(f, z_i) = \text{real-valued scalar}
\end{equation}  

The different real-valued scalar represent the error, which we aim to minimize
by adjusting the parameters/weights of some decision function $f$.
At the end of the training, $f$ is supposed to behave like the unknown process $P$.

%There are common strategies to choose this loss function.

   \item The question is how to adjust the parameters/weights that represent the
   state function of $f$? 
   
   There are two approaches
   \begin{enumerate}
     \item super-vised learning - Sect.\ref{sec:supervised-learning}
     \item unsupervised learning - Sect.\ref{sec:unsupervised-learning}
   \end{enumerate} 
\end{enumerate}

\subsection{supervised learning}
\label{sec:supervised-learning}

Given the training samples, we always know the expected outcome, i.e. 
training samples always come in pair, $z_i=(x,y)$ with x=input,
y=target-output/decision.  NOTE: For multiple data samples, we use upper-case
symbol, i.e. $Z=(X,Y)$ - a vector/matrix.

QUESTION: minimize the difference $f(x)-y$, which is typically represented as,
depending on the type of values of $Y$.
\begin{enumerate}
  \item {\bf regression} form: if $Y$ is a real-value scalar, or vector.
  
  \begin{equation}
L(f,(X,Y)) = ||f(X)-Y||^2
\end{equation}
  
  
  \item {\bf classification} form: if $Y$ is an integer-value scalar, or a
  symbol) corresponding to a class index, i.e. category index.
  
 $f(X)$ represents a probability function, with $f_i(X)$ returns the probability
 that X falls into category $i$-th. So, the contraints are
 \begin{equation}
 \begin{split}
 \sum_i f_i(X) = 1 \\
 f_i(Y) \ge 0
 \end{split}
 \end{equation}
  
 This maps to the Bayesian decision rule: $P(Y=i | X)$. When the system
 'predict'/estimate the right category, it means that the probability for that
 should be high (as close to 1 as possible). This maps to the loss function by
 converting into the \verb!-log()! function
 \begin{equation}
 L(f, (X,Y)) = -log f_Y(X)
 \end{equation}
  
  
\end{enumerate}




\subsection{unsupervised learning}
\label{sec:unsupervised-learning}

As we don't know the target of individual data item/point, one way to find out
which data points belong to the same 'unknown' target is by 'visualizing' them
in the space, and find the clustering, i.e. those clustered together are
supposed to be of the same 'unknown' target.

This is known as {\bf density estimation}, i.e. discovering the hidden
properties inside the data. So, if we can estimate its density function
(Sect.\ref{sec:density-estimation}), i.e. we can have a good chance of
predicting the 'target' of a new data. Density estimation walks the line between
unsupervised learning, feature engineering, and data modeling.

% Consider the {\bf unknown} process $P$ whose outcome/output as a random
% variable, i.e. we don't know it's next value, but we may know its probbility
% density function, if the output is known to be clustered following certain
% distribution pattern.

 
\section{Density estimation}
\label{sec:density-estimation}

Density estimation is a very simple concept being used in unsupervised learing
(Sect.\ref{sec:density-estimation}), and most people are already familiar with
one common density estimation technique: the histogram
(Sect.\ref{sec:histogram}).


\section{Classification}
\label{sec:classification}

Classification systems play an important role in a wide variety of fields by
classifying the available information based on some criteria.
Many decision-making tasks are instances of classification problem or can be
easily formulated into a classifica- tion problem, e.g., prediction and
forecasting tasks, diagnosis tasks, and pattern recognition.

In supervised classification, the classifier is constructed from patterns in a
training set and assign label for unseen patterns from a test set. Each pattern
belongs to one class, and the number of classes is known.

Techniques:
\begin{enumerate}
  \item support vector machine (SVM)

Lu, S.-X., Wang, X.-Z.: A comparison among four SVM classification methods:
LSVM, NLSVM, SSVM and NSVM. In: Proc. of 2004 Int'l on Machine Learning
and Cybernetics, vol. 7, pp. 4277-4282 (2004)

  
  \item artificial neural network (ANN) -
  Sect.\ref{sec:ANN-articial-neuron-network}
  
  \item underlying probability densities

Duda, R.O., Hart, P.E., Stork, D.G.: Pattern classification. John Wiley \& Sons,
New York (2001)

\end{enumerate}

\section{Data sets}

Every instance in any dataset used by machine learning algorithms is represented
using the same set of features. The features may be continuous, categorical or
binary.


\subsection{Collection}

The first step is collecting the dataset. If a requisite expert is available,
then s/he could suggest which fields (attributes, features) are the most
informative. If not, then the simplest method is that of "brute-force," which
means measuring everything available in the hope that the right (informative,
relevant) features can be isolated.
However, a dataset collected by the "brute-force" method is not directly
suitable for induction. It contains in most cases noise and missing feature
values, and therefore requires significant pre-processing (Zhang et al., 2002)
- Sect.\ref{sec:data-preprocessing}.

\subsection{Pre-processing}
\label{sec:data-preprocessing}	


The second step is the data preparation and data preprocessiong.
Depending on the circumstances, researchers have a number of methods to choose
from to handle missing data (Batista \& Monard, 2003). Hodge \& Austin (2004)
have recently introduced a survey of contemporary techniques for outlier (noise)
detection.
 
\section{-- outlier detections}

Many schemes for outlier detection
have proposed by researchers.

\begin{enumerate}
  \item  The early outlier detection methods are based-distribution

  \item However, in practice the distribution is not easily to known.

  \item  distance-based: Knorr E.M et al. (1999)
  
Knorr, E.M., Ng, R.T.: Finding Intensional Knowledge of Distance-based Outliers.
In: Proc. 25th Int'l Conference on very large Data Bases, Edinburgh, Scotland, pp.
211-222 (1999)

CONS: can not work well when the data set does not have uniform density global

  \item multilayer perceptron (MLP) and variations of the RBF network: Liu-Gader
  (2000)
  
Liu, J., Gader, P.: Outlier Rejection with MLPs and Variants of RBF Networks.
In: Proc. 15th Int'l Conference on Pattern Recognition, vol. 2, pp. 680-683 (2000)

\end{enumerate}



\section{Generalized Linear Model (GLM) using Logistic regression}
\label{sec:GLM}
\label{sec:Gradient-Boosting}

Applications:
\begin{enumerate}
  \item customer churn prediction
\end{enumerate}

\section{Gradient Boosting Machines (GBM) using Decision trees}
\label{sec:GBM}

Applications:
\begin{enumerate}
  \item airline flight delay prediction
\end{enumerate}


\chapter{Machine Learning on Distributed Systems}
\label{chap:machine_learning_distributed_system}

MapReduce is a two-stage approach (Map first, then Reduce) for solving many
query-based problem on very large dataset. Hadoop's MapReduce impelement is
disk-based. 

\section{Mahout}
\label{sec:mahout}

Mahout (from 2009) aims to produce free implementation of distributed (or
scalable) machine learning algorithms focused primarily on 
\begin{enumerate}
  \item batch-based collaborative filtering
  \item clustering
  \item classification
  \item recommendation
\end{enumerate}
Many of the implementations in Mahout use Apache Hadoop's disk-based Map-Reduce
paradigm. However, it does not restrict contributions only for Hadoop-based
implementation. Contributions that run on a single node or on a  non-Hadoop
cluster are also welcomed. 

Recently, Mahout community is switching to  a better approach, using in-memory
(cache-based) approach, proves a many times better in performance (Spark -
Sect.\ref{sec:apache_spark}, H2O or Flink).




\url{http://en.wikipedia.org/wiki/Apache_Mahout}


\section{Apache Spark}
\label{sec:apache_spark}

Apache Spark provides an implementation to many machine learning algorithms to
read data stored on HDFS file system (Sect.\ref{sec:HadoopFS}) or Cassandra
database system (Sect.\ref{chap:cassandra}). 
Spark use in-memory approach, i.e. data is cached, so it is considered faster
than Apache Mahout (Sect.\ref{sec:mahout}).

Spark joined incubator since June 2013, and graduated as Apache top-level
project in Feb-2014. Spark provides API in Java, Scala and Python, which has
proved to scale well on 2000 nodes (Amazon EC2 research lab) and 1000 nodes (on
production). It provides implementation for many machine learning and graph
processing algorithm.



