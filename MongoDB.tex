\chapter{MongoDB}
\label{chap:MongoDB}

MongoDB is a new, open-source NoSQL database system. MongoDB can be used as a
file system, taking advantage of load balancing and data replication features
over multiple machines for storing files.  This is known as {\bf Sharding}
(Sect.\ref{sec:sharding}).

MongoDB is very much similar to CouchDB (Chap.\ref{chap:MongoDB}), except it has
a single-master node only. Data is stored in JSON-like format with dynamic
schemas (which is called BSON format).
To access data in MongoDB from Hadoop's MapReduce jobs (including from Apache
Hive and Pig), including incrementally writing output back to collections in
MongoDB, we can use {\bf MongoDB Hadoop Connector}
(Sect.\ref{sec:MongoDB_Hadoop-Connector}).
\url{http://www.mongodb.com/hadoop-and-mongodb}


MongoDB was first released in 2007 by 10gen (now MongoDB Inc.) and shifted to
open-source in 2009; while MongoDB Inc. offers commercial supports. 

MongoDB does not have a version control  system like CouchDB. So,
it uses readers-writer lock that allows concurrent read access to a database but
exclusive write access to a single write operation.

Riak and Cassandra are both implementations of Amazon's Dynamo, which could each
do a good job of that.
\url{http://wiki.basho.com/Riak-Compared-to-Cassandra.html}

All MongoDB nodes are readable, but only a master is writable.
About a single point of failure: MongoDB uses replicasets for distributing reads
and sharding for distributing writes. Note: MongoDB does not support
multi-master replication. To have multi-master database, we can use Apache
Cassandra, which has no single point of failure. \url{http://hadoop.apache.org/}

HBase is a scalable, distributed database that supports structured data storage
for large tables. HBase can do very big, can do concurrent writes on various
nodes. Its design is good for many attributes on each document/record.
CouchDB has good multiple-write conflict support but with a simpler document
space.
\url{http://stackoverflow.com/questions/5436202/looking-for-distributed-scalable-database-solution-where-all-nodes-are-read-writ}

\section{Sharding}
\label{sec:sharding}

Sharding is the process of storing data records across multiple machines and is
MongoDB's approach to meeting the demands of data growth.  With sharding, you
add more machines to support data growth and the demands of read and write
operations. Each machine is called a {\bf Shard}, e.g. Shard A, Shard B, \ldots

Each shard processes fewer operations as the cluster grows. As a result, a
cluster can increase capacity and throughput horizontally.
\url{http://docs.mongodb.org/manual/core/sharding-introduction/}

\subsection{Install}

A sharded cluster consists of shards, config servers, and mongos instances.

There is a daemon \verb!mongod! instance running on each Shard machine.
The config server holds the metadata about the cluster, e.g. shard location of
the data. 

\url{http://docs.mongodb.org/manual/tutorial/deploy-shard-cluster/}




% \section{MongoDB on HDFS}

% HDFS is not schema-based; data of any type can be stored.
% Hadoop jobs define a schema for reading the data within the scope of the job.
% Hadoop does not use indexes. Data is scanned for each query.
% Hadoop jobs tend to execute over several minutes.

% As Hadoop with MapReduce jobs is batch processing, i.e. not designed for
% real-time application, we can use it in health-care when data and processing
% tasks that exceed the capacity of a single server, and applications tend to be
% based on large volumes of data. For real-time processing, we can use Apache
% HBase on HDFS. Several information is stored in log files in Hadoop.
% Apache Flume is a tool for collecting data from log files into HDFS.


\section{MongoDB Hadoop Connector}
\label{sec:MongoDB_Hadoop-Connector}

Instead of using HDFS for the file system, Hadoop's MapReduce can
access data on MongoDB using MongoDB Hadoop Connector. 

\url{http://docs.mongodb.org/ecosystem/tools/hadoop/}


\url{http://www.ikanow.com/how-well-does-mongodb-integrate-with-hadoop/}


