\chapter{Large-scale Graph Data}
\label{chap:GraphData_LargeScale}

Despite its prominent role in big data analytics, MapReduce
(Sect.\ref{sec:MapReduce}) is not the optimal programming model for graph
processing. From the graph-processing point of view, the basic MapReduce
programming model is inadequate because most graph algorithms are iterative and
traverse the graph in some way. Thus, Hadoop and its associated technologies
(such as Pig and Hive) were not designed mainly to support scalable processing
of graph-structured data.
\url{http://www.ibm.com/developerworks/library/os-giraph/}

\begin{mdframed}
Efforts to extend Hadoop for graph-based data include:
Surfer and GBASE.
\end{mdframed}

\section{Introduction}

\subsection{Graphs}

Graphs are abstract data structures that model structural relationships among
objects. It use graph structure with nodes, edges and properties to represent and store
data.
\begin{itemize}
  \item nodes: represent entities (like a table in RDBMS)
  
  Each entity has properties
  
  \item properties: pertinent information that relate to nodes and edges
  
  \item edges: lines that connect nodes to nodes and represent the relationship
  
  \item faster than RDBMS for associative data sets
  
  \item scale more naturally to large dataset as they do not require expensive
  join operations
\end{itemize}

In the most common sense, a graph is an ordered pair $G=(V,E)$ comprising a set $V$ 
of {\bf vertices} or {\bf nodes}, together with a set $E$ of {\bf edges} or {\bf lines}.
$V$ and $E$ are usually taken to be finite, and many well-known results are not true for infinite graphs.
\begin{itemize}
  \item the {\bf order} of a graph is the number of vertices, |$V$|.
  \item the {\bf size} of a graph is the number of edges, |$E$|.
  \item the {\bf degree of a vertex} is the number of edges that connect to it. NOTE:
  an edge that connect to the vertex at both ends (a loop) is counted twice.
\end{itemize}

\subsection{Graphs and Linear Algebra}

The adjacency matrix of an unidirected graph is a (0,1)-matrix, after all. This
makes linear algebra a natural language and tool to deal with graph.

A graph ()

\subsection{Applications}

Many practical problems can be represented by graphs.
They are now widely used for data modeling in application domains for which
identifying relationship patterns, rules, and anomalies is useful.

Applications that use graph-based representation of data include:
web graph, social networks, the Semantic Web, knowledge bases, protein-protein
interaction networks, and bibliographical networks, among many others.

\begin{itemize}
  \item Computer Science: networks of communication, data organization, computational devices, flow of computation.
  
\begin{itemize}
  \item Web graph: Google estimates that the total number of web pages exceeds 1 trillion;
experimental graphs of the World Wide Web contain more than 20 billion nodes
(pages) and 160 billion edges (hyperlinks).
  
  \item Social networks: Facebook reportedly consists of more than a billion
  users (nodes) and more than 140 billion friendship relationships (edges) in
  2012. The LinkedIn network contains almost 8 million nodes and 60 million
  edges.
  
  \item Semantic Web:  the ontology of DBpedia (derived from Wikipedia),
  currently contains 3.7 million objects (nodes) and 400 millions facts (edges).
\end{itemize}
  
  \item Chemistry: natural model for a molecule (vertices = atoms, edges=bonds)
  and can be used iin computer processing of molecular structures (e.g. chemical editors, database searching)
  
\end{itemize}

 
Hence, the efficiency of graph computations depends heavily on interprocessor
bandwidth as graph structures are sent over the network iteration after
iteration.  To implement iterative programs, programmers might manually issue
multiple MapReduce jobs and orchestrate their execution with a driver program.
This manual orchestration of an iterative program in MapReduce has two key
problems: 
\begin{enumerate}
  \item if much data is unchanged, reload and reprocess at each iteration wastes
  I/O, network bandwidth, and processor resources.
  
  \item synchronization again increases resources in terms of scheduling extra
  tasks, reading extra data from disks, and moving data across network.
\end{enumerate}


Examples:
\begin{enumerate}
  \item Neo4j
  \item OrientDB
  \item HyperGraphDB
  \item Titan: distributed, real-time, scalable 
  \item GraphBase
  
\end{enumerate}

There are different graph models:
\begin{itemize}
  \item RDF
  \item Property Graph
  
  \item Labeled and Directed Attributed Multigraph:
  
  \item Attributed Multigraph

  \item Mixed, Framework-managed Simple Graph
  
  \item Object-oriented multi-relational labeled hypergraph
  
  \item Dynamically typed, object-oriented graph, multigraph, semantic models
  
  \item Triplestore
\end{itemize}


\section{Apache Giraph}
\label{sec:giraph}

Giraph, a distributed and fault-tolerant system that adopts the {\it Bulk
Synchronous Parallel} (BSP) programming model to run parallel algorithms for
processing large-scale graph data.



\section{GraphLab}
\label{sec:GraphLab}

GraphLab, a graph-based, high-performance, distributed computation framework
that is written in C++


\section{GraphX}
\label{sec:GraphX}


GraphX is built on top of Sparks (Spark SQL, Spark Streaming), MLlib (machine
learning).

\url{https://spark.apache.org/graphx/}

\section{Tinkerpop}
\label{sec:Tinkerpop}